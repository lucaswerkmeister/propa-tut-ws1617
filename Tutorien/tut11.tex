\documentclass{beamer}
\usepackage{tut}

\def\tuttitle{Java Parallelprogrammierung}
\date{2017-01-30/31}

\begin{document}
\normalsize
\normalem

\lstset{language=Java}

\begin{frame}[plain]
  \titlepage
\end{frame}

\begin{frame}
  \frametitle{Organisatorisches}
  Nächste Woche müssen zwei Übungsblätter besprochen werden;
  sie werden auf zwei Tage verteilt:
  \begin{itemize}
  \item Am \textbf{Montag, 14:00 Uhr, Raum 131} geht es um \emph{Parallelprogrammierung};
  \item am \textbf{Dienstag, 11:30 Uhr, Raum 301} geht es um \emph{Compilerbau}.
  \end{itemize}
\end{frame}

\begin{frame}
  \frametitle{Amdahlsches Gesetz}
  Gegeben ist ein Pool mit $N$ Threads.
  Davon sind \SI{90}{\percent} Leser, welche gleichzeitig auf einem Puffer arbeiten können,
  und \SI{10}{\percent} Schreiber, welche exklusiven Zugriff auf den Puffer benötigen.
  Jeder Leser benötigt \SI{2}{\second} und jeder Schreiber \SI{3}{\second} zur Ausführung.
  Wo liegt gemäß Amdahls Gesetz die obere Grenze für den Speedup dieses Programms auf einem 4-Kern-Prozessor?
  \pause
  
  \begin{align*}
    S &= \frac{1}{\frac{P}{N} + (1-P)} & P&\text{: parallelisierbarer Anteil} \\
    \onslide<3->{
      P &= \frac{\num{2} \cdot \num{0.9}}{\num{2} \cdot \num{0.9} + \num{3} \cdot \num{0.1}} = \frac{\num{6}}{\num{7}} \approx \num{0.86} & N &= 4 \\
      S &= \num{2.8}
    }
  \end{align*}
\end{frame}

\begin{frame}
  \frametitle{Paralleler Primzahltest mit Futures}
  Gegeben ist Java-Code, der sequenziell einfache Primzahltests durchführt und die Anzahl gefundener Primzahlen bestimmt.
  Zur Reduktion der Laufzeit und zur Auslastung von Mehrkernsystemen soll er parallelisiert werden,
  indem jeder Thread einen Teil des Intervalls autonom prüft und die Anzahl gefundener Primzahlen zurückgibt.
  Dabei soll das Future-Konzept verwendet werden.
\end{frame}

\begin{frame}[fragile]
  \frametitle{\lstinline{DivisionSeq}}
  \lstinputlisting[linerange={3-3,5-5,7-12,14-19,21-25}]{tut11/primes/DivisionSeq.java}
\end{frame}

\begin{frame}
  \frametitle{Mergesort}
  Mergesort ist ein \emph{teile-und-herrsche}-Algorithmus und kann gut mit dem Fork/Join-Pattern implementiert werden.
  Die zu sortierende Liste wird rekursiv in kleinere Listen zerlegt,
  bis eine einzelne Liste einfach sortiert werden kann (z.\,B. einelementig).
  Anschließend fügt man die Teile wieder in der richtigen Reihenfolge zusammen.
  
  Mergesort soll mit einem \lstinline{ForkJoinPool} in Java implementiert werden.
\end{frame}

\begin{frame}
  \frametitle{Barriere}
  Implementiere eine zyklische Barriere gemäß diesem Interface:
  \lstinputlisting[linerange={3-6}]{tut11/barrier/IBarrier.java}
  Einziger Konstruktorparameter ist die Anzahl an Threads,
  die \lstinline{await} aufrufen müssen,
  um die Barriere freizugeben.
  \lstinline{freeAll} gibt die Barriere manuell frei;
  alle aktuell wartenden Threads werden freigegeben
  und alle, die danach \lstinline{await} aufrufen,
  müssen wieder warten.
\end{frame}

\begin{frame}
  \frametitle{Urlaubsreservierungssystem}
  Es soll ein System entwickelt werden,
  in dem Mitarbeiter einer Firma Urlaub reservieren können.
  Mitarbeiter sind in Buddy-Pärchen eingeteilt;
  mindestens ein Mitarbeiter eines Paares muss immer arbeiten
  (d.\,h., wenn ein Mitarbeiter Urlaub reserviert,
  muss der Buddy in dieser Zeit arbeiten).
  Der Einfachkeit halber seien Namen von Mitarbeitern eindeutig und Urlaub wird nur als ganzer Monat angefragt.
  
  Aufgabe ist es, die Methode \lstinline{VacationReservationSystem.tryTakeVacation} zu implementieren.
  Für die ersten Aufgabenteile darf der restliche Code nicht verändert werden.
\end{frame}

\begin{frame}
  \frametitle{\lstinline{synchronized}}
  Implementiere \lstinline{VacationReservationSystem.tryTakeVacation}.
  Versehe dabei die ganze Methode mit dem \lstinline{synchronized}-Modifikator.

  Was ist der Nachteil dieser Implementierung?
  
  \pause
  Diese Synchronisierung ist sehr grobgranular und kann zum Performanz-Engpass werden.
\end{frame}

\begin{frame}
  \frametitle{\sout{\lstinline{synchronized}}}
  Was kann passieren, wenn man \lstinline{synchronized} weglässt? Warum?
  
  \pause
  Es kann passieren, dass beide Mitarbeiter eines Buddy-Paars gleichzeitig Urlaub reservieren.
\end{frame}

\begin{frame}
  \frametitle{Monitore}
  Implementiere feingranularere Synchronisierung, indem \lstinline{EmployeeVacationRecord}s als Monitor-Objekte verwendet werden.
  Betrete zuerst den Monitor für den aufrufenden Mitarbeiter und dann den für den Buddy.
  Was passiert? Wieso?
  
  \pause
  Deadlock, wenn beide Mitarbeiter eines Buddy-Paars den eigenen Monitor betreten und dann auf dem Buddy blockieren.
\end{frame}

\begin{frame}
  \frametitle{Monitore: Lösung}
  Modifiziere diese Lösung, sodass kein Deadlock mehr auftritt.
  
  Was sind die Coffman-Bedingungen?
  \pause
  \begin{description}
  \item[Mutual Exclusion]
    Eine Ressource kann nur durch einen einzigen Prozess gleichzeitig verwendet werden;
    nur ein Prozess darf sich gleichzeitig in der \emph{critical section} befinden.
  \item[Hold And Wait]
    Eine Prozess darf eine Ressource anfragen und dabei, während er darauf wartet, dass sie von einem anderen Prozess freigegeben wird, weitere Ressourcen halten.
  \item[No Preemption]
    Eine Ressource kann nur freiwillig durch den Prozess selbst freigegeben werden.
  \item[Circular Wait]
    Im Graph der Prozess-Ressourcen-Abhängigkeiten besteht ein Kreis.
  \end{description}
  
  \pause
  Sortiere die Mitarbeiter des Buddy-Paars alphabetisch nach Name
  und betrete den ersten Mitarbeiter zuerst.
  Dadurch entsteht kein \emph{Circular Wait}.
\end{frame}

\begin{frame}
  \frametitle{Optimale Lösung}
  Zum Schluss darf der gesamte Quelltext verwendet werden.
  Wie könnte eine noch bessere Lösung aussehen?
  
  \pause
  Zum Beispiel: pro Buddy-Pärchen nur ein Monitor,
  der im \lstinline{EmployeeVacationRecord} hinterlegt wird.
\end{frame}

\begin{frame}
  \frametitle{Aktoren}
  \emph{Aktoren} sind ein alternatives Modell zur Parallelprogrammierung.
  Anstatt mehrere \emph{Threads} mit \emph{shared state} zu haben
  (Zustand – d.\,h. Variablen – auf den mehrere Threads zugreifen können),
  hat man \emph{Aktoren}, die einander Nachrichten schicken und auf Nachrichten reagieren.
\end{frame}

\begin{frame}
  \frametitle{Design by Contract}
  Bei \emph{Design by Contract} beschreibt eine Methode,
  welche \emph{Vorbedingungen} sie benötigt, um korrekt zu funktionieren
  (zum Beispiel: \lstinline{someParameter != null}, \lstinline{!someList.isEmpty()}),
  und welche \emph{Nachbedingungen} sie in diesem Fall selbst garantiert.
  
  Einige formale Beweissysteme,
  etwa das u.\,a. am KIT entwickelte \emph{KeY},
  können beweisen, dass der Code einer Methode ihre Verträge tatsächlich erfüllt.
  Durch Kombination der Verträge aller Methoden lässt sich dann beweisen,
  dass ein ganzes Programm so funktioniert wie gewünscht.
\end{frame}

\end{document}
