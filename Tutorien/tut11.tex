\documentclass{beamer}
\usepackage{tut}

\def\tuttitle{Java Parallelprogrammierung}
\date{2017-01-30/31}

\begin{document}
\normalsize
\normalem

\lstset{language=Java}

\begin{frame}[plain]
  \titlepage
\end{frame}

\begin{frame}
  \frametitle{Amdahlsches Gesetz}
  Gegeben ist ein Pool mit $N$ Threads.
  Davon sind \SI{90}{\percent} Leser, welche gleichzeitig auf einem Puffer arbeiten können,
  und \SI{10}{\percent} Schreiber, welche exklusiven Zugriff auf den Puffer benötigen.
  Jeder Leser benötigt \SI{2}{\second} und jeder Schreiber \SI{3}{\second} zur Ausführung.
  Wo liegt gemäß Amdahls Gesetz die obere Grenze für den Speedup dieses Programms auf einem 4-Kern-Prozessor?
  \pause
  
  \begin{align*}
    S &= \frac{1}{\frac{P}{N} + (1-P)} & P&\text{: parallelisierbarer Anteil} \\
    \onslide<3->{
      P &= \frac{\num{2} \cdot \num{0.9}}{\num{2} \cdot \num{0.9} + \num{3} \cdot \num{0.1}} = \frac{\num{6}}{\num{7}} \approx \num{0.86} & N &= 4 \\
      S &= \num{2.8}
    }
  \end{align*}
\end{frame}

\begin{frame}
  \frametitle{Paralleler Primzahltest mit Futures}
  Gegeben ist Java-Code, der sequenziell einfache Primzahltests durchführt und die Anzahl gefundener Primzahlen bestimmt.
  Zur Reduktion der Laufzeit und zur Auslastung von Mehrkernsystemen soll er parallelisiert werden,
  indem jeder Thread einen Teil des Intervalls autonom prüft und die Anzahl gefundener Primzahlen zurückgibt.
  Dabei soll das Future-Konzept verwendet werden.
\end{frame}

\begin{frame}[fragile]
  \frametitle{\lstinline{DivisionSeq}}
  \lstinputlisting[linerange={3-3,5-5,7-12,14-19,21-25}]{tut11/primes/DivisionSeq.java}
\end{frame}

\begin{frame}
  \frametitle{Mergesort}
  Mergesort ist ein \emph{teile-und-herrsche}-Algorithmus und kann gut mit dem Fork/Join-Pattern implementiert werden.
  Die zu sortierende Liste wird rekursiv in kleinere Listen zerlegt,
  bis eine einzelne Liste einfach sortiert werden kann (z.\,B. einelementig).
  Anschließend fügt man die Teile wieder in der richtigen Reihenfolge zusammen.
  
  Mergesort soll mit einem \lstinline{ForkJoinPool} in Java implementiert werden.
\end{frame}

\end{document}
