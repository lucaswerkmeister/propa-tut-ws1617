\documentclass{beamer}
\usepackage{tut}

\def\tuttitle{Lambda-Kalkül, Typprüfung}
\date{2016-11-28/29}

\begin{document}
\normalsize
\normalem

\begin{frame}[plain]
  \titlepage
\end{frame}

\begin{frame}
  \frametitle{Church-Zahlen}
  \begin{align*}
    0 &= 0 \\
    1 &= 1 + 0 \\
    2 &= 1 + 1 + 0 \\
    3 &= 1 + 1 + 1 + 0 \\
    \\
    c_0 &= λs.~λz.~z \\
    c_1 &= λs.~λz.~s~z \\
    c_2 &= λs.~λz.~s~s~z \\
    c_3 &= λs.~λz.~s~s~s~z
  \end{align*}
  Die Church-Zahl für $n∈ℕ$ besteht daraus, dass die Nachfolgerfunktion~$s$ $n$~mal auf das Nullelement~$z$ angewendet wird.
\end{frame}

\begin{frame}[fragile]
  \frametitle{Church-Zahlen in Haskell}
  Der Typ einer Church-Zahl in Haskell ist:
  \begin{lstlisting}
    type Church t = (t -> t) -> t -> t
    ----               s        z
  \end{lstlisting}
  Definiere
  \begin{lstlisting}
    int2church :: Integer -> Church t
    church2int :: Church Integer -> Integer
  \end{lstlisting}
  um von und zu gewöhnlichen Zahlen zu konvertieren.
  (\lstinline{church2int} geht ohne Rekursion.)
\end{frame}

\begin{frame}[fragile]
  \frametitle{\lstinline{int2church}}
  \begin{align*}
    c_0 &= λs.~λz.~z \\
    c_n &= λs.~λz.~\underbrace{s~(s~(\ldots~s}_{\text{$n$ mal}}~z))
  \end{align*}
  \begin{lstlisting}
    int2church :: Integer -> Church t
  \end{lstlisting}
  \pause
  \begin{lstlisting}
    int2church 0 = \s z -> z
    int2church n = \s z -> int2church (n-1) s (s z)
    -- oder
    int2church n = \s z -> s (int2church (n-1) s z)
  \end{lstlisting}
\end{frame}

\begin{frame}[fragile]
  \frametitle{\lstinline{church2int}}
  \begin{lstlisting}
    church2int :: Church Integer -> Integer
  \end{lstlisting}
  \pause
  \begin{lstlisting}
    church2int c = c (1+) 0
  \end{lstlisting}
  Damit ist
  \begin{lstlisting}
    church2int c3
      = (\s z ->  s   ( s   ( s     z)))  (1+)  0
      =          (1+) ((1+) ((1+)   0))
      =           1  +  1  +  1  +  0
      = 3
  \end{lstlisting}
\end{frame}

\begin{frame}[fragile]
  \frametitle{Redexe, Auswertungsstrategie}
  Markiere alle Redexe.
  Welcher wird unter Normal(en)reihenfolge, Call-By-Name und Call-By-Value jeweils zuerst reduziert?
  \pause
  \begin{align*}
    t_1 &= (\cul<3->[c1]{λt.}~λf.~f)~\cul<3->[c1]{((\cul<3->[c2]{λy.}~(\cul<3->[c3]{λx.}~x~x)~\cul<3->[c3]{(λx.~x~x)})~\cul<3->[c2]{((\cul<3->[c4]{λx.}~x)~\cul<3->[c4]{(λx.~x)})})}~(λt.~λf.~f)
  \end{align*}
  \pause[4]%
  Normalreihenfolge und Call-By-Name werten \textcolor{c1}{blau} zuerst aus,
  Call-By-Value wertet \textcolor{c4}{rot} zuerst aus.
  \pause
  \begin{align*}
    t_2 &= λy.~(\cul<6->[c1]{λz.}~(\cul<6->[c2]{λx.}~x)~\cul<6->[c2]{(λx.~x)}~z)~\cul<6->[c1]{y}
  \end{align*}
  \pause[7]%
  Normalenreihenfolge wertet \textcolor{c1}{blau} zuerst aus,
  Call-By-Value und Call-By-Name machen nichts
  ($t_2$ ist schon ein Wert).
\end{frame}

\end{document}
