\documentclass{beamer}
\usepackage{tut}

\def\tuttitle{Typinferenz}
\date{2017-01-09/10}

\begin{document}
\normalsize
\normalem

\begin{frame}[plain]
  \titlepage
\end{frame}

\begin{frame}
  \frametitle{Typinferenz Vorgehen}
  Vier Schritte zur Typinferenz:
  \begin{enumerate}
  \item Herleitungsbaum erstellen, mit frischen Typvariablen $α_i$ überall
  \item Gleichungssystem $C$ für die $α_i$ extrahieren gemäß Typisierungsregeln
  \item Unifikator $σ_C$ finden, der $C$ löst
  \item Typ $σ_C(α_1)$ bestimmen
  \end{enumerate}
  Schritt 1 und 2 macht man am besten zusammen.
\end{frame}

\begin{frame}
  \frametitle{Typinferenz Aufgabe}
  Gegeben seien folgende $λ$-Terme:
  \begin{align*}
    t_1 &= λ?z.~?z \\
    t_2 &= λ?f.~λ?x.~?f~?x \\
    t_3 &= λ?f.~λ?x.~?f~(?f~?x) \\
    t_4 &= λ?x.~λ?y.~?y~(?x~?y)
  \end{align*}
  Es soll für jeden Term eine Typinferenz durchgeführt werden.
\end{frame}

\newcommand{\spickzettel}[0]{
  \begin{frame}
    \frametitle{Typinferenz Spickzettel}
    \begin{multicols}{2}
      \begin{prooftree}
        \AxiomC{$c ∈ \lambdaval{Const}$}
        \rulename{Const}
        \UnaryInfC{$Γ \vdash c : α_i$}
      \end{prooftree}
      \break
      $α_i = τ_c$
    \end{multicols}
    
    \begin{multicols}{2}
      \begin{prooftree}
        \AxiomC{$(Γ, x : α_i)(x) = α_j$}
        \rulename{Var}
        \UnaryInfC{$(Γ, x : α_i) \vdash x : α_j$}
      \end{prooftree}
      \break
      $α_i = α_j$
    \end{multicols}
    
    \begin{multicols}{2}
      \begin{prooftree}
        \AxiomC{$Γ, x : α_j \vdash t : α_k$}
        \rulename{Abs}
        \UnaryInfC{$Γ \vdash λ x.~t : α_i$}
      \end{prooftree}
      \break
      $α_i = α_j → α_k$
    \end{multicols}
    
    \begin{multicols}{2}
      \begin{prooftree}
        \AxiomC{$Γ \vdash t_1 : α_k$}
        \AxiomC{$Γ \vdash t_2 : α_j$}
        \rulename{App}
        \BinaryInfC{$Γ \vdash t_1~t_2 : α_i$}
      \end{prooftree}
      \break
      $α_k = α_j → α_i$
    \end{multicols}
    
    \begin{enumerate}
    \item Herleitungsbaum erstellen, mit frischen Typvariablen $α_i$ überall
    \item Gleichungssystem $C$ für die $α_i$ extrahieren gemäß Typisierungsregeln
    \item Unifikator $σ_C$ finden, der $C$ löst
    \item Typ $σ_C(α_1)$ bestimmen
    \end{enumerate}
  \end{frame}
}

\spickzettel

\begin{frame}
  \frametitle{Typinferenz $t_1$}
  \[t_1 = λ?z.~?z\]
  \pause
  \begin{multicols}{2}
    \begin{prooftree}
      \AxiomC{$(?z:α_2)(?z)=α_3$}
      \rulename[c2]{Var}
      \UnaryInfC{$?z:α_2 \vdash ?z:α_3$}
      \rulename[c1]{Abs}
      \UnaryInfC{$\vdash λ?z.~?z : α_1$}
    \end{prooftree}
    \break
    \begin{align*}
      C &= \{{\color{c1}α_1=α_2→α_3}, {\color{c2}α_2=α_3}\} \\
      σ_C &= [α_1 ⇒ α_2 → α_2, α_3 ⇒ α_2] \\
      σ_C(α_1) &= α_2 → α_2
    \end{align*}
  \end{multicols}
\end{frame}

\spickzettel

\begin{frame}
  \frametitle{Typinferenz $t_2$}
  \[t_2 = λ?f.~λ?x.~?f~?x\]
  \pause
  \begin{prooftree}
    \AxiomC{$(?f:α_2,?x:α_4)(?f)=α_6$}
    \rulename[c4]{Var}
    \UnaryInfC{$?f:α_2,?x:α_4 \vdash ?f:α_6$}
    \AxiomC{$(?f:α_2,?x:α_4)(?x)=α_7$}
    \rulename[c5]{Var}
    \UnaryInfC{$?f:α_2,?x:α_4 \vdash ?x:α_7$}
    \rulename[c3]{App}
    \BinaryInfC{$?f:α_2,?x:α_4 \vdash ?f~?x: α_5$}
    \rulename[c2]{Abs}
    \UnaryInfC{$?f:α_2 \vdash λ?x.~?f~?x:α_3$}
    \rulename[c1]{Abs}
    \UnaryInfC{$\vdash λ?f.~λ?x.~?f~?x:α_1$}
  \end{prooftree}
  \begin{align*}
    C = &\left\{
      \begin{array}{@{}l}
        {\color{c1}α_1=α_2→α_3}, {\color{c2}α_3=α_4→α_5}, \\
        {\color{c3}α_6=α_7→α_5}, {\color{c4}α_2=α_6}, {\color{c5}α_4=α_7}
      \end{array}
      \right\} \\
    σ_C = &\left[
      \begin{array}{@{}l}
        α_1 ⇒ (α_7→α_5)→α_7→α_5, α_2 ⇒ α_7→α_5, \\
        α_3 ⇒ α_7→α_5, α_4 ⇒ α_7, α_6 ⇒ α_7→α_5
      \end{array}
      \right] \\
    σ_C(α_1) = &(α_7→α_5)→α_7→α_5
  \end{align*}
\end{frame}

\spickzettel

\begin{frame}
  \frametitle{Typinferenz $t_3$}
  \[t_3 = λ?f.~λ?x.~?f~(?f~?x)\]
  \pause
  \begin{tiny}\let\rulenamesize=\tiny
    \begin{prooftree}
      \AxiomC{$(?f:α_2,?x:α_4)(?f)=α_6$}
      \rulename[c4]{Var}
      \UnaryInfC{$?f:α_2,?x:α_4 \vdash ?f:α_6$}
      \AxiomC{$(?f:α_2,?x:α_4)(?f)=α_8$}
      \rulename[c6]{Var}
      \UnaryInfC{$?f:α_2,?x:α_4 \vdash ?f:α_8$}
      \AxiomC{$(?f:α_2,?x:α_4)(?x)=α_9$}
      \rulename[c7]{Var}
      \UnaryInfC{$?f:α_2,?x:α_4 \vdash ?x:α_9$}
      \rulename[c5]{App}
      \BinaryInfC{$?f:α_2,?x:α_4 \vdash ?f~?x:α_7$}
      \rulename[c3]{App}
      \BinaryInfC{$?f:α_2,?x:α_4 \vdash ?f~(?f~?x): α_5$}
      \rulename[c2]{Abs}
      \UnaryInfC{$?f:α_2 \vdash λ?x.~?f~(?f~?x):α_3$}
      \rulename[c1]{Abs}
      \UnaryInfC{$\vdash λ?f.~λ?x.~?f~(?f~?x):α_1$}
    \end{prooftree}
  \end{tiny}
  \begin{align*}
    C = &\left\{
      \begin{array}{@{}l}
        {\color{c1}α_1=α_2→α_3}, {\color{c2}α_3=α_4→α_5}, \\
        {\color{c3}α_6=α_7→α_5}, {\color{c4}α_2=α_6}, \\
        {\color{c5}α_8=α_9→α_7}, {\color{c6}α_2=α_8}, {\color{c7}α_4=α_9}
      \end{array}
      \right\} \\
    σ_C = &\left[
      \begin{array}{@{}l}
        α_1 ⇒ (α_9→α_9)→α_9→α_9, α_2 ⇒ α_9→α_9, \\
        α_3 ⇒ α_9→α_9, α_4 ⇒ α_9, α_5 ⇒ α_9, \\
        α_6 ⇒ α_9→α_9, α_7 ⇒ α_9, α_8 ⇒ α_9→α_9
      \end{array}
      \right] \\
    σ_C(α_1) = &(α_9→α_9)→α_9→α_9
  \end{align*}
\end{frame}

\spickzettel

\begin{frame}
  \frametitle{Typinferenz $t_4$}
  \[t_4 = λ?x.~λ?y.~?y~(?x~?y)\]
  \pause
  \begin{tiny}\let\rulenamesize=\tiny
    \begin{prooftree}
      \AxiomC{$(?x:α_2,?y:α_4)(?y)=α_6$}
      \rulename[c4]{Var}
      \UnaryInfC{$?x:α_2,?y:α_4 \vdash ?y:α_6$}
      \AxiomC{$(?x:α_2,?y:α_4)(?x)=α_8$}
      \rulename[c6]{Var}
      \UnaryInfC{$?x:α_2,?y:α_4 \vdash ?x:α_8$}
      \AxiomC{$(?x:α_2,?y:α_4)(?y)=α_9$}
      \rulename[c7]{Var}
      \UnaryInfC{$?x:α_2,?y:α_4 \vdash ?y:α_9$}
      \rulename[c5]{App}
      \BinaryInfC{$?x:α_2,?y:α_4 \vdash ?x~?y:α_7$}
      \rulename[c3]{App}
      \BinaryInfC{$?x:α_2,?y:α_4 \vdash ?y~(?x~?y): α_5$}
      \rulename[c2]{Abs}
      \UnaryInfC{$?x:α_2 \vdash λ?y.~?y~(?x~?y):α_3$}
      \rulename[c1]{Abs}
      \UnaryInfC{$\vdash λ?x.~λ?y.~?y~(?x~?y):α_1$}
    \end{prooftree}
  \end{tiny}
  \begin{align*}
    C = &\left\{
      \begin{array}{@{}l}
        {\color{c1}α_1=α_2→α_3}, {\color{c2}α_3=α_4→α_5}, \\
        {\color{c3}α_6=α_7→α_5}, {\color{c4}α_4=α_6}, \\
        {\color{c5}α_8=α_9→α_7}, {\color{c6}α_2=α_8}, {\color{c7}α_4=α_9}
      \end{array}
      \right\} \\
    σ_C = &\left[
      \begin{array}{@{}l}
        α_1 ⇒ ((α_7→α_5)→α_7)→(α_7→α_5)→α_5, \\
        α_2 ⇒ (α_7→α_5)→α_7, α_3 ⇒ (α_7→α_5)→α_5, \\
        α_4 ⇒ α_7→α_5, α_6 ⇒ α_7→α_5, \\
        α_8 ⇒ (α_7→α_5)→α_7, α_9 ⇒ α_7→α_5
      \end{array}
      \right] \\
    σ_C(α_1) = &((α_7→α_5)→α_7)→(α_7→α_5)→α_5
  \end{align*}
\end{frame}

\begin{frame}
  \frametitle{Typabstraktion}
  In der Typabstraktion $\mathit{ta}(τ,Γ)$ werden nicht \emph{alle} freien Typvariablen von $τ$ quantifiziert,
  sondern nur die, die nicht frei in den Typannahmen $Γ$ vorkommen.
  
  Überlegen Sie anhand des $λ$-Terms $λ?x.~\Let ?y=?x \In ?y~?x$, was passiert,
  wenn man diese Beschränkung aufhebt!
  \pause
  
  Ohne diese Beschränkung lässt sich für den gesamten Ausdruck der Typ $α→β$ nachweisen,
  indem man im $\texttt{\textbf{in}}$-Teil das fälschlicherweise allquantifizierte $y:∀α.α$ mit $α→β$ instanziiert.
  Tatsächlich reduziert der Ausdruck aber zu einer Selbstapplikation,
  die bekanntlich nicht typisierbar ist –
  der Typ kann also nicht stimmen!
\end{frame}

\end{document}
