\documentclass{beamer}
\usepackage{tut}

\def\tuttitle{Haskell-Grundlagen, Listen}
\date{2016-10-28/31}

\begin{document}
\normalsize
\normalem

\begin{frame}[plain]
  \titlepage
\end{frame}

\begin{frame}
  \frametitle{Nachtrag zum Im-Browser-Haskell}
  \begin{itemize}
  \item sollte jetzt stabiler sein
  \item am besten Notebooks selbst herunterfahren, wenn sie nicht mehr benötigt werden
  \item insbesondere: wenn eine lange Berechnung nicht zu Ende kommt, nicht einfach das Notebook schließen!
  \end{itemize}
\end{frame}

\begin{frame}[fragile]
  \frametitle{\lstinline{pow1}}
  Schreiben Sie eine rekursive Funktion \lstinline{pow1}, die Basis \lstinline{b} und Exponent \lstinline{e} als Parameter nimmt und $b^e$ naiv über
  \begin{align*}
    b^0 &= 1\\
    b^{e+1} &= b \cdot b^e
  \end{align*}
  berechnet.
  \pause
  \begin{lstlisting}
    pow1 b 0 = 1
    pow1 b e = b * pow1 b (e-1)
  \end{lstlisting}
\end{frame}

\begin{frame}[fragile]
  \frametitle{\lstinline{pow2}}
  Wesentlich effizienter ist es, bei jedem Rekursionsschritt den Exponenten zu halbieren und die Basis zu quadrieren:
  \begin{align*}
    b^{2e} &= (b^2)^e \\
    b^{2e+1} &= b \cdot (b^2)^e
  \end{align*}
  Schreiben Sie weitere Funktion \lstinline{pow2}, die Potenz auf diese Weise effizienter berechnet.
  Wie viele Aufrufe braucht \lstinline{pow2} im Vergleich zu \lstinline{pow1}?
  \pause
  \begin{lstlisting}
    pow2 b 0 = 1
    pow2 b e
      | even e    = pow2 (b*b) (e `div` 2)
      | otherwise = b * pow2 (b*b) (e `div` 2)
  \end{lstlisting}
  
  \pause
  \lstinline{pow1} läuft in $\Theta(e)$, \lstinline{pow2} in $\Theta(\log e)$.
\end{frame}

\begin{frame}[fragile]
  \frametitle{\lstinline{pow3}}
  Transformieren Sie \lstinline{pow2} in endrekursive Version \lstinline{pow3} (Hilfsfunktion mit Akkumulator).
  Außerdem: Fehlerbehandlung bei negativem Exponenten mittels \lstinline{error}.
  \pause
  \begin{lstlisting}
    pow3 b e
      | e < 0     = error "Negativer Exponent"
      | otherwise = pow3Acc b e 1 where
          pow3Acc b e acc
            | e == 0    = acc
            | even e    = pow3Acc (b*b) (e `div` 2) acc
            | otherwise = pow3Acc (b*b) (e `div` 2) (b*acc)
  \end{lstlisting}
\end{frame}

\begin{frame}[fragile]
  \frametitle{\lstinline{root}}
  Implementieren Sie die Funktion \lstinline{root e r}, die ganzzahlige, \lstinline{e}-te Wurzel von \lstinline{r} berechnet,
  also die größte nichtnegative ganze Zahl $x$ mit $x^e \leq r$.
  
  Verwendetes Verfahren: Intervallhalbierung.
  Hilfsfunktion erhält Grenzen $a$, $b$ eines Intervalls mit $a \leq x < b$.
  Ist $b-a=1$, so ist $a=x$.
  Sonst halbiere das Intervall und prüfe, in welcher Hälfte die Zahl liegt.
  
  Vorsicht bei Randfällen!
  Für welche \lstinline{e}, \lstinline{r} ist Berechnung möglich?
  \pause
  \begin{lstlisting}
    root e r
      | e < 1     = error "Zu kleiner Wurzelexponent"
      | r < 0     = error "Negativer Radikand"
      | otherwise = searchRoot 0 (r+1) where
          searchRoot a b
            | b - a <= 1       = a
            | pow3 half e <= r = searchRoot half b
            | otherwise        = searchRoot a half
            where half = (a + b) `div` 2
  \end{lstlisting}
\end{frame}

\begin{frame}[fragile]
  \frametitle{\lstinline{isPrime}}
  Schreiben Sie eine Funktion \lstinline{isPrime}, die für $n \geq 2$ testet, ob $n$ durch eine Zahl zwischen 2 und $\sqrt{2}$ teilbar ist.
  \pause
  \begin{lstlisting}
    isPrime n
      | n < 2     = error "Zu kleine Zahl bei Primzahltest"
      | otherwise = searchPrime 2 where
          limit = root 2 n
          searchPrime d
            | d > limit      = True
            | n `mod` d == 0 = False
            | otherwise      = searchPrime (d+1)
  \end{lstlisting}
  \pause
  \begin{lstlisting}
    isPrime n = not $ any (\k -> n `mod` k == 0) [2..root 2 n]
  \end{lstlisting}
  \pause
  \begin{lstlisting}
    isPrime n = null $ filter (\k -> n `mod` k == 0) [2..root 2 n]
  \end{lstlisting}
\end{frame}

\begin{frame}[fragile]
  \frametitle{\lstinline{insert}}
  Schreiben Sie eine Funktion \lstinline{insert}, die eine Zahl in eine aufsteigend sortierte Liste von Zahlen korrekt einfügt.
  \pause
  \begin{lstlisting}
    insert x [] = [x]
    insert x (y:ys)
      | x <= y    = x:y:ys
      | otherwise = y : insert x ys
  \end{lstlisting}
  Kann man auch in endrekursiver Form schreiben, bringt aber nicht viel.
\end{frame}

\begin{frame}[fragile]
  \frametitle{\lstinline{insertSort}}
  Verwenden Sie \lstinline{insert} für Funktion \lstinline{insertSort}, die Liste von ganzen Zahlen sortiert.
  \pause
  \begin{lstlisting}
    insertSort [] = []
    insertSort (x:xs) = insert x (insertSort xs)
  \end{lstlisting}
  \pause
  Für Profis, die voll funktionale Variante:
  \pause
  \begin{lstlisting}
    insertSort = foldr insert []
  \end{lstlisting}
  Große Magie! Ab zur Tafel.
\end{frame}

\begin{frame}[fragile]
  \frametitle{\lstinline{merge}}
  Schreiben Sie eine Funktion \lstinline{merge}, die zwei sortierte Listen zu einer sortieren Liste zusammenführt.
  \pause
  \begin{lstlisting}
    merge xs [] = xs
    merge [] ys = ys
    merge (x:xs) (y:ys)
      | x <= y    = x : merge xs (y:ys)
      | otherwise = y : merge (x:xs) ys
  \end{lstlisting}
\end{frame}

\begin{frame}[fragile]
  \frametitle{\lstinline{mergeSort}}
  Verwenden Sie \lstinline{merge} für Funktion \lstinline{mergeSort}, die Liste von ganzen Zahlen sortiert.
  \pause
  \begin{lstlisting}
    mergeSort [] = []
    mergeSort [x] = [x]
    mergeSort xs = merge
      (mergeSort $ take half xs)
      (mergeSort $ drop half xs)
      where half = length xs `div` 2
  \end{lstlisting}
\end{frame}

\end{document}
