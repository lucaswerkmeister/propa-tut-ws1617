\documentclass{beamer}
\usepackage{tut}

\def\tuttitle{Haskell-Grundlagen, Listen}
\date{2016-10-28/31}

\begin{document}
\normalsize
\normalem

\begin{frame}[plain]
  \titlepage
\end{frame}

\begin{frame}
  \frametitle{Nachtrag zum Im-Browser-Haskell}
  \begin{itemize}
  \item sollte jetzt stabiler sein
  \item am besten Notebooks selbst herunterfahren, wenn sie nicht mehr benötigt werden
  \item insbesondere: wenn eine lange Berechnung nicht zu Ende kommt, nicht einfach das Notebook schließen!
  \end{itemize}
\end{frame}

\begin{frame}[fragile]
  \frametitle{\lstinline{pow1}}
  Schreiben Sie eine rekursive Funktion \lstinline{pow1}, die Basis \lstinline{b} und Exponent \lstinline{e} als Parameter nimmt und $b^e$ naiv über
  \begin{align*}
    b^0 &= 1\\
    b^{e+1} &= b \cdot b^e
  \end{align*}
  berechnet.
  \pause
  \begin{lstlisting}
    pow1 b 0 = 1
    pow1 b e = b * pow1 b (e-1)
  \end{lstlisting}
\end{frame}

\end{document}
