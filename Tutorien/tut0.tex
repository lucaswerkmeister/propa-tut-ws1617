\documentclass{beamer}
\usepackage{tut}

\def\tuttitle{Erste Schritte in Haskell}
\date{2016-10-24/25}

\begin{document}
\normalsize
\normalem

\begin{frame}[plain]
  \titlepage
\end{frame}

\begin{frame}
  \frametitle{Ziele des Tutoriums}
  \pause
  \begin{itemize}
  \item vergangenes Blatt besprechen
  \item auf nächstes Blatt vorbereiten
  \item Stoff wiederholen, vertiefen
  \item Fragen beantworten, helfen
  \item \sout{Insiderwissen über die Klausur weitergeben}
  \end{itemize}
\end{frame}

\begin{frame}
  \frametitle{Abgabe der Übungsblätter}
  \begin{itemize}
  \item E-Mail an \href{mailto:lucas.werkmeister@student.kit.edu?subject=[PP2016] Blatt X, MN 1234567}{lucas.werkmeister@student.kit.edu}
  \item handschriftliche Abgaben in den Briefkasten
  \item E-Mail-Abgaben werden per E-Mail zurückgeschickt
  \item handschriftliche Abgaben werden im Tutorium zurückgegeben
  \end{itemize}
  (Eine Abgabe reicht, Programme müssen nicht noch zusätzlich in den Briefkasten.)
\end{frame}

\begin{frame}
  \frametitle{Interaktives Tutorium}
  \begin{itemize}
  \item \url{https://jupyter.lucaswerkmeister.de}
  \item Haskell-System im Browser
  \item Verwendung: „Notebook“ erstellen und mit Shift+Enter an den Interpreter abschicken
  \item Außerdem: Bluetooth-Tastatur zum Herumreichen
  \item (Bitte keinen Unfug treiben)
  \end{itemize}
\end{frame}

\begin{frame}[fragile]
  \frametitle{\lstinline{max3}}
  \begin{itemize}
  \item \lstinline{if .. then .. else}
    \pause
    \begin{lstlisting}
      max3if x y z = if (x<=y)
                     then if (y<=z) then z else y
                     else if (x<=z) then z else x
    \end{lstlisting}
    \pause
  \item \emph{guard}-Notation
    \pause
    \begin{lstlisting}
      max3guard x y z =
        | x>=y && x>=z = x
        | y>=x && y>=z = y
        | otherwise    = z
    \end{lstlisting}
    \pause
  \item \lstinline{max}
    \pause
    \begin{lstlisting}
      max3max x y z = max x (max y z)
      -- oder
      max3max x y z = max x $ max y z
    \end{lstlisting}
  \end{itemize}
\end{frame}

\begin{frame}[fragile]
  \frametitle{\lstinline{max3}-Test}
  \begin{lstlisting}
    testMax3 :: (Integer -> Integer -> Integer -> Integer) ->
      Integer -> Integer -> Integer -> Bool
    testMax3 max3 x y z = let max = max3 x y z in
      max >= x && max >= y && max >= z &&
      (max == x || max == y || max == z)
  \end{lstlisting}
  \pause
  Test mit QuickCheck:
  \begin{lstlisting}
    import Test.QuickCheck
    quickCheck (testMax3 max3if)
  \end{lstlisting}
  (Typsignatur auf \lstinline{testMax3} ist nötig, sonst testet QuickCheck mit dem trivialen Typ/Wert \lstinline{()}.)
\end{frame}

\begin{frame}[fragile]
  \frametitle{\lstinline{max3}-Test}
  \lstinline{testMax3} sieht noch etwas unhandlich aus.
  Mit funktionaler Programmierung geht das besser!
  \pause
  \begin{lstlisting}
    testMax3 max3 x y z = let max = max3 x y z in
      all (max>=) [x,y,z] &&
      elem max [x,y,z]
  \end{lstlisting}
  \begin{itemize}
  \item \lstinline{all} prüft, ob ein Prädikat (Funktion von Element auf \lstinline{Bool}) für alle Elemente einer Liste erfüllt ist
  \item \lstinline{>=} wird unterversorgt, ergibt Funktion, die prüft, ob \lstinline{max} größer oder gleich dem Argument ist
  \item \lstinline{elem} prüft, ob ein Element in einer Liste enthalten ist
  \end{itemize}
\end{frame}

\begin{frame}
  \frametitle{Endrekursion}
  Rekursive Funktionsaufrufe können optimiert werden,
  wenn die Funktion nach dem rekursiven Aufruf nichts mehr macht,
  außer das Ergebnis des Aufrufs selbst zurückzugeben.
  Eine Funktion in dieser Form nennt man \emph{endrekursiv}.
  
  Deshalb \emph{kann} es sich für \emph{bestimmte} performanzkritische Funktionen lohnen,
  sie in endrekursive Form zu bringen,
  üblicherweise mithilfe von zusätzlichen „Akkumulator“-Parametern.
  
  In allen anderen Fällen
  (etwa in der Klausur, wenn eine endrekursive Form nicht explizit gefordert ist)
  riskiert man damit jedoch nur Fehler.
\end{frame}

\end{document}
