\documentclass{beamer}
\usepackage{tut}

\def\tuttitle{Aktoren und Design by Contract}
\date{2017-02-06/07}

\begin{document}
\normalsize
\normalem

\lstset{language=Java}

\begin{frame}[plain]
  \titlepage
\end{frame}

\begin{frame}
  \frametitle{Einfacher Nachrichtenaustausch}
  Erstelle eine von \lstinline{UntypedActor} erbende Klasse \lstinline{Kid},
  die auf Schimpfwort-Nachrichten (Strings) eine genervte Antwort auf die Konsole schreibt
  und ab der vierten Nachricht nach der Mutter ruft.
  
  Verwendung (Beispiel):
  \lstinputlisting[linerange={7-20}]{tut12/simpsons/src/main/java/de/lucaswerkmeister/simpsons/TeaseLisa.java}
\end{frame}

\begin{frame}
  \frametitle{\lstinline{Kid}}
  \lstinputlisting[linerange={5-20}]{tut12/simpsons/src/solution/java/de/lucaswerkmeister/simpsons/Kid.java}
\end{frame}

\begin{frame}
  \frametitle{Speisende Philosophen}
  Fünf Philosophen sitzen um einen Tisch herum,
  jeder mit einem Teller.
  Zwischen zwei Tellern liegt jeweils eine Gabel.
  Jeder Philosoph ist stets entweder mit Denken oder mit Essen beschäftigt.
  Zum Essen benötigt er zwei Gabeln, die links und rechts von ihm liegen.
  Zum Denken legt er diese wieder zurück.
  
  Was ist der Vorteil des Aktor-Konzepts gegenüber Threads,
  insbesondere für dieses Problem?
  
  \pause
  Eine Thread-Implementierung dieses Problems ist anfällig für Deadlocks,
  die mit Aktoren nicht auftreten können.
\end{frame}

\begin{frame}
  \frametitle{Speisende Philosophen in Akka}
  Implementiere das Problem mit Akka Aktoren in Java.
  Erstelle Aktoren für den Tisch und die Philosophen
  und implementiere geeignete Kommunikation.
  Essen und Denken besteht darin, ein zufälliges Zeitintervall zu warten (\lstinline{Thread.sleep()}).
  Als Abbruchbedingung des Programms kann ein einfacher Timeout dienen.
\end{frame}

\begin{frame}[fragile]
  \frametitle{Einkaufen}
  Der folgende Code beschreibt ein einfaches Einkaufsszenario.
  Güter (\lstinline{Good}s) werden aus einer Ladentheke (\lstinline{Counter})
  von Personen (\lstinline{Person}s) in einen Einkaufswagen (\lstinline{Cart}) gelegt.
  Die zugehörige Methode \lstinline{put} ist mit einem Vertrag versehen,
  der Vor- und Nachbedingungen als Java-Ausdrücke angibt;
  dabei bezieht sich \lstinline{\old(AUSDRUCK)}
  auf den Wert von \lstinline{AUSDRUCK} zu Beginn des Aufrufs.
  
  Wo und durch wen wird der Vertrag verletzt?
\end{frame}

\begin{frame}[fragile]
  \frametitle{\lstinline{Good}, \lstinline{Counter}}
  \begin{lstlisting}
    public interface Good {}
    
    public interface Counter {
        /**
         * Gibt ein Gut aus der Theke zurueck.
         * Gibt null zurueck, falls sie leer ist.
         */
        public Good takeSomeGood();
    }
  \end{lstlisting}
\end{frame}

\begin{frame}[fragile]
  \frametitle{\lstinline{Cart}}
  \begin{lstlisting}
    public class Cart {
        private Set<Good> goods = new HashSet<Good>();
        /**
         * Vorbedingungen:
         * good != null
         * 
         * Nachbedingungen:
         * goods.containsAll(\old(goods))
         * goods.contains(good)
         * goods.size() == \old(goods).size() + 1
         */
        public void put(Good good) {
            goods.add(good);
        }
        public Collection<Good> getGoods() {
            return goods;
        }
    }
  \end{lstlisting}
\end{frame}

\begin{frame}[fragile]
  \frametitle{\lstinline{Person}}
  \begin{lstlisting}
    public class Person {
        public void shop(Counter counter) {
            Cart cart = new Cart();
            for (int i = 0;
                 i < new Random().nextInt(20);
                 i++) {
                cart.put(counter.takeSomeGood());
            }
        }
    }
  \end{lstlisting}
\end{frame}

\begin{frame}[fragile]
  \frametitle{Vertragsverletzung\onslide<2->{en}}
  \begin{itemize}
  \item
    \lstinline{Person} verletzt den Vertrag.
    \lstinline{Counter.takeSomeGood()} kann \lstinline{null} zurückgeben,
    der Vertrag von \lstinline{Cart.put} verlangt allerdings \lstinline{good != null}.
    \lstinline{Person} muss prüfen, ob das von der Theke genommene Gut auch existiert.
    \pause
  \item
    \lstinline{Cart} verletzt den Vertrag.
    Die \lstinline{goods} sind ein \lstinline{Set};
    wird ein \lstinline{Good} zweimal hinzugefügt,
    so bleibt \lstinline{goods} mit dem zweiten Aufruf unverändert,
    und die Größe bleibt gleich,
    was die Nachbedingung \lstinline{goods.size() == \old(goods).size() + 1} verletzt.
  \end{itemize}
\end{frame}

\end{document}
