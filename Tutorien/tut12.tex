\documentclass{beamer}
\usepackage{tut}

\def\tuttitle{Aktoren und Design by Contract}
\date{2017-02-06/07}

\begin{document}
\normalsize
\normalem

\lstset{language=Java}

\begin{frame}[plain]
  \titlepage
\end{frame}

\begin{frame}
  \frametitle{Einfacher Nachrichtenaustausch}
  Erstelle eine von \lstinline{UntypedActor} erbende Klasse \lstinline{Kid},
  die auf Schimpfwort-Nachrichten (Strings) eine genervte Antwort auf die Konsole schreibt
  und ab der vierten Nachricht nach der Mutter ruft.
  
  Verwendung (Beispiel):
  \lstinputlisting[linerange={7-20}]{tut12/simpsons/src/main/java/de/lucaswerkmeister/simpsons/TeaseLisa.java}
\end{frame}

\begin{frame}
  \frametitle{\lstinline{Kid}}
  \lstinputlisting[linerange={5-20}]{tut12/simpsons/src/solution/java/de/lucaswerkmeister/simpsons/Kid.java}
\end{frame}

\begin{frame}[fragile]
  \frametitle{Ping Pong}
  Implementiere einen Aktor \lstinline{PingPong},
  der mit einem Aktor und einem Namen initialisiert wird
  und auf \lstinline{Integer}-Nachrichten reagiert,
  indem er \lstinline{"<name> received <wert>"} ausgibt
  und dann den empfangenen Wert um 1 erhöht
  und an den im Konstruktor übergebenen Aktor weiterleitet.
  (Ist kein Aktor gesetzt,
  so soll an den Absender der empfangenen Nachricht zurückgeschickt werden.)
  
  Signatur des Konstruktors:
  \begin{lstlisting}
    PingPong(ActorRef sendTo, String name)
  \end{lstlisting}
\end{frame}

\begin{frame}
  \frametitle{Speisende Philosophen}
  Fünf Philosophen sitzen um einen Tisch herum,
  jeder mit einem Teller.
  Zwischen zwei Tellern liegt jeweils eine Gabel.
  Jeder Philosoph ist stets entweder mit Denken oder mit Essen beschäftigt.
  Zum Essen benötigt er zwei Gabeln, die links und rechts von ihm liegen.
  Zum Denken legt er diese wieder zurück.
  
  Was ist der Vorteil des Aktor-Konzepts gegenüber Threads,
  insbesondere für dieses Problem?
  
  \pause
  Eine Thread-Implementierung dieses Problems ist anfällig für Deadlocks,
  die mit Aktoren nicht auftreten können.
\end{frame}

\begin{frame}
  \frametitle{Speisende Philosophen in Akka}
  Implementiere das Problem mit Akka Aktoren in Java.
  Erstelle Aktoren für den Tisch und die Philosophen
  und implementiere geeignete Kommunikation.
  Essen und Denken besteht darin, ein zufälliges Zeitintervall zu warten (\lstinline{Thread.sleep()}).
  Als Abbruchbedingung des Programms kann ein einfacher Timeout dienen.
\end{frame}

\begin{frame}[fragile]
  \frametitle{Einkaufen}
  Der folgende Code beschreibt ein einfaches Einkaufsszenario.
  Güter (\lstinline{Good}s) werden aus einer Ladentheke (\lstinline{Counter})
  von Personen (\lstinline{Person}s) in einen Einkaufswagen (\lstinline{Cart}) gelegt.
  Die zugehörige Methode \lstinline{put} ist mit einem Vertrag versehen,
  der Vor- und Nachbedingungen als Java-Ausdrücke angibt;
  dabei bezieht sich \lstinline{\old(AUSDRUCK)}
  auf den Wert von \lstinline{AUSDRUCK} zu Beginn des Aufrufs.
  
  Wo und durch wen wird der Vertrag verletzt?
\end{frame}

\begin{frame}[fragile]
  \frametitle{\lstinline{Good}, \lstinline{Counter}}
  \begin{lstlisting}
    public interface Good {}
    
    public interface Counter {
        /**
         * Gibt ein Gut aus der Theke zurueck.
         * Gibt null zurueck, falls sie leer ist.
         */
        public Good takeSomeGood();
    }
  \end{lstlisting}
\end{frame}

\begin{frame}[fragile]
  \frametitle{\lstinline{Cart}}
  \begin{lstlisting}
    public class Cart {
        private Set<Good> goods = new HashSet<Good>();
        /**
         * Vorbedingungen:
         * good != null
         * 
         * Nachbedingungen:
         * goods.containsAll(\old(goods))
         * goods.contains(good)
         * goods.size() == \old(goods).size() + 1
         */
        public void put(Good good) {
            goods.add(good);
        }
        public Collection<Good> getGoods() {
            return goods;
        }
    }
  \end{lstlisting}
\end{frame}

\begin{frame}[fragile]
  \frametitle{\lstinline{Person}}
  \begin{lstlisting}
    public class Person {
        public void shop(Counter counter) {
            Cart cart = new Cart();
            for (int i = 0;
                 i < new Random().nextInt(20);
                 i++) {
                cart.put(counter.takeSomeGood());
            }
        }
    }
  \end{lstlisting}
\end{frame}

\begin{frame}[fragile]
  \frametitle{Vertragsverletzung\onslide<2->{en}}
  \begin{itemize}
  \item
    \pause
    \lstinline{Person} verletzt den Vertrag.
    \lstinline{Counter.takeSomeGood()} kann \lstinline{null} zurückgeben,
    der Vertrag von \lstinline{Cart.put} verlangt allerdings \lstinline{good != null}.
    \lstinline{Person} muss prüfen, ob das von der Theke genommene Gut auch existiert.
    \pause
  \item
    \lstinline{Cart} verletzt den Vertrag.
    Die \lstinline{goods} sind ein \lstinline{Set};
    wird ein \lstinline{Good} zweimal hinzugefügt,
    so bleibt \lstinline{goods} mit dem zweiten Aufruf unverändert,
    und die Größe bleibt gleich,
    was die Nachbedingung \lstinline{goods.size() == \old(goods).size() + 1} verletzt.
  \end{itemize}
\end{frame}

\begin{frame}
  \frametitle{Personalverwaltung}
  Es soll eine Software zur Personalverwaltung eines Unternehmens entwickelt werden.
  Bereits vorgegeben ist eine \lstinline{Employee}-Klasse
  mit den Methoden \lstinline{hire()}, \lstinline{fire()} und \lstinline{isEmployed()}.
  Nun soll folgendes Interface implementiert werden:
  \lstinputlisting[linerange={3-7}]{tut12/company/Company.java}
  Anforderungen:
  \begin{itemize}
  \item
    Ein Mitarbeiter darf nur bei einer Firma angestellt sein.
  \item
    Es darf nur versucht werden, einen Mitarbeiter einzustellen, wenn er noch nicht angestellt ist.
  \item
    Es darf nur versucht werden, einen Mitarbeiter zu feuern, wenn er angestellt ist.
  \end{itemize}
\end{frame}

\begin{frame}[fragile]
  \frametitle{\lstinline{hire}-Vertrag}
  Schreibe Vertrag für \lstinline{hire}
  (Java-Ausdrücke, \lstinline{\old(AUSDRUCK)} für \lstinline{AUSDRUCK} zu Beginn der Methode).
  \begin{itemize}
  \item Vorbedingungen:\pause
    \begin{lstlisting}
      employee != null
      !getEmployees().contains(employee)
      !employee.isEmployed()
    \end{lstlisting}
  \item Nachbedingungen:\pause
    \begin{lstlisting}
      employee.isEmployed()
      getEmployees().size == \old(getEmployees()).size() + 1
      getEmployees().containsAll(\old(getEmployees()))
      getEmployees().contains(employee)
    \end{lstlisting}
  \end{itemize}
\end{frame}

\begin{frame}[fragile]
  \frametitle{\lstinline{fire}-Vertrag}
  Schreibe Vertrag für \lstinline{fire}
  (Java-Ausdrücke, \lstinline{\old(AUSDRUCK)} für \lstinline{AUSDRUCK} zu Beginn der Methode).
  \begin{itemize}
  \item Vorbedingungen:\pause
    \begin{lstlisting}
      employee != null
      getEmployees().contains(employee)
      employee.isEmployed()
    \end{lstlisting}
  \item Nachbedingungen:\pause
    \begin{lstlisting}
      !employee.isEmployed()
      getEmployees().size == \old(getEmployees()).size() - 1
      \old(getEmployees()).containsAll(getEmployees())
      !getEmployees().contains(employee)
    \end{lstlisting}
  \end{itemize}
\end{frame}

\begin{frame}
  \frametitle{Implementierung}
  Implementiere das \lstinline{Company}-Interface
  und stelle dabei den Vertrag mit \lstinline{assert}-Statements sicher.
  
  Was sind Vorteile und Nachteile der Vertragsbeschreibung mit \lstinline{assert}s
  gegenüber Sprachen wie Object Constraint Language (OCL) oder Java Modeling Language (JML)?
\end{frame}

\end{document}
