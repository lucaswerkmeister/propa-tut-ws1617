\documentclass{beamer}
\usepackage{tut}

\def\tuttitle{Aktoren und Design by Contract}
\date{2017-02-06/07}

\begin{document}
\normalsize
\normalem

\lstset{language=Java}

\begin{frame}[plain]
  \titlepage
\end{frame}

\begin{frame}
  \frametitle{Einfacher Nachrichtenaustausch}
  Erstelle eine von \lstinline{UntypedActor} erbende Klasse \lstinline{Kid},
  die auf Schimpfwort-Nachrichten (Strings) eine genervte Antwort auf die Konsole schreibt
  und ab der vierten Nachricht nach der Mutter ruft.
  
  Verwendung (Beispiel):
  \lstinputlisting[linerange={7-20}]{tut12/simpsons/src/main/java/de/lucaswerkmeister/simpsons/TeaseLisa.java}
\end{frame}

\begin{frame}
  \frametitle{\lstinline{Kid}}
  \lstinputlisting[linerange={5-20}]{tut12/simpsons/src/solution/java/de/lucaswerkmeister/simpsons/Kid.java}
\end{frame}

\begin{frame}
  \frametitle{Speisende Philosophen}
  Fünf Philosophen sitzen um einen Tisch herum,
  jeder mit einem Teller.
  Zwischen zwei Tellern liegt jeweils eine Gabel.
  Jeder Philosoph ist stets entweder mit Denken oder mit Essen beschäftigt.
  Zum Essen benötigt er zwei Gabeln, die links und rechts von ihm liegen.
  Zum Denken legt er diese wieder zurück.
  
  Was ist der Vorteil des Aktor-Konzepts gegenüber Threads,
  insbesondere für dieses Problem?
  
  \pause
  Eine Thread-Implementierung dieses Problems ist anfällig für Deadlocks,
  die mit Aktoren nicht auftreten können.
\end{frame}

\begin{frame}
  \frametitle{Speisende Philosophen in Akka}
  Implementiere das Problem mit Akka Aktoren in Java.
  Erstelle Aktoren für den Tisch und die Philosophen
  und implementiere geeignete Kommunikation.
  Essen und Denken besteht darin, ein zufälliges Zeitintervall zu warten (\lstinline{Thread.sleep()}).
  Als Abbruchbedingung des Programms kann ein einfacher Timeout dienen.
\end{frame}

\end{document}
