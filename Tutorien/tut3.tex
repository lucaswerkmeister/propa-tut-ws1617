\documentclass{beamer}
\usepackage{tut}

\def\tuttitle{Laziness, Typsystem}
\date{2016-11-14/15}

\begin{document}
\normalsize
\normalem

\begin{frame}[plain]
  \titlepage
\end{frame}

\begin{frame}[fragile]
  \frametitle{\lstinline{fibs}}
  Definiere unendliche Liste \lstinline{fibs :: [Integer]} aller Fibonacci-Zahlen:
  \begin{lstlisting}
    [0, 1, 1, 2, 3, 5, 8, 13, 21, 34, ...]
  \end{lstlisting}
  Lässt sich mit \lstinline{zipWith} sehr kompakt definieren.
  \pause
  \begin{lstlisting}
    fibs = 0 : 1 : zipWith (+) fibs (tail fibs)
  \end{lstlisting}
  Alternativ, ohne \lstinline{zipWith}:
  \begin{lstlisting}
    fibs = fibonacci 0 1
      where fibonacci x y = x : fibonacci y (x+y)
  \end{lstlisting}
  \lstinline{zipWith} war ein \emph{Hinweis}, nicht Teil der Aufgabenstellung selbst, daher wäre auch diese Lösung in der Klausur gültig gewesen.
\end{frame}

\begin{frame}[fragile]
  \frametitle{\lstinline{collatz}}
  Die Collatz-Vermutung besagt, dass die Zahlenfolge
  \[a_{n+1} = \begin{cases}\frac{a_n}{2} & \text{falls $a_n$ gerade} \\ 3a_n + 1 & \text{sonst}\end{cases}\]
  für jeden Startwert $a_0 ∈ ℕ$ die 1 erreicht.
  
  Definiere Funktion \lstinline{collatz}, so dass \lstinline{collatz a0} die unendliche Liste aller Folgenglieder berechnet.
  Verwende \lstinline{iterate} mit geeigneter Hilfsfunktion.
  \pause
  \begin{lstlisting}
    collatz a0 = iterate step a0
      where step an
              | even an   = an `div` 2
              | otherwise = 3*an + 1
  \end{lstlisting}
\end{frame}

\end{document}
