\documentclass{beamer}
\usepackage{tut}

\def\tuttitle{Laziness, Typsystem}
\date{2016-11-14/15}

\begin{document}
\normalsize
\normalem

\begin{frame}[plain]
  \titlepage
\end{frame}

\begin{frame}[fragile]
  \frametitle{\lstinline{fibs}}
  Definiere unendliche Liste \lstinline{fibs :: [Integer]} aller Fibonacci-Zahlen:
  \begin{lstlisting}
    [0, 1, 1, 2, 3, 5, 8, 13, 21, 34, ...]
  \end{lstlisting}
  Lässt sich mit \lstinline{zipWith} sehr kompakt definieren.
  \pause
  \begin{lstlisting}
    fibs = 0 : 1 : zipWith (+) fibs (tail fibs)
  \end{lstlisting}
  Alternativ, ohne \lstinline{zipWith}:
  \begin{lstlisting}
    fibs = fibonacci 0 1
      where fibonacci x y = x : fibonacci y (x+y)
  \end{lstlisting}
  \lstinline{zipWith} war ein \emph{Hinweis}, nicht Teil der Aufgabenstellung selbst, daher wäre auch diese Lösung in der Klausur gültig gewesen.
\end{frame}

\begin{frame}[fragile]
  \frametitle{\lstinline{collatz}}
  Die Collatz-Vermutung besagt, dass die Zahlenfolge
  \[a_{n+1} = \begin{cases}\frac{a_n}{2} & \text{falls $a_n$ gerade} \\ 3a_n + 1 & \text{sonst}\end{cases}\]
  für jeden Startwert $a_0 ∈ ℕ$ die 1 erreicht.
  
  Definiere Funktion \lstinline{collatz}, so dass \lstinline{collatz a0} die unendliche Liste aller Folgenglieder berechnet.
  Verwende \lstinline{iterate} mit geeigneter Hilfsfunktion.
  \pause
  \begin{lstlisting}
    collatz a0 = iterate step a0
      where step an
              | even an   = an `div` 2
              | otherwise = 3*an + 1
  \end{lstlisting}
\end{frame}

\begin{frame}[fragile]
  \frametitle{\lstinline{num}}
  Definiere Funktion \lstinline{num}, die für Startwert $a_0$ das kleinste $n$ mit $a_n = 1$ bestimmt.
  Testfall: \lstinline{num 4 == 2}, denn $a_0=4,a_1=2,a_2=1$.
  \pause
  \begin{lstlisting}
    num m = length $ takeWhile (/=1) $ collatz m
  \end{lstlisting}
  Oder:
  \begin{lstlisting}
    import Data.List -- for elemIndex
    import Data.Maybe -- for fromJust
    num m = fromJust $ elemIndex 1 $ collatz m
  \end{lstlisting}
\end{frame}

\begin{frame}[fragile]
  \frametitle{\lstinline{maxNum}}
  Definiere Funktion \lstinline{maxNum}.
  \lstinline{maxNum a b} soll ein Tupel \lstinline{(m, num m)} liefern, sodass \lstinline{num m} maximal ist für alle $m ∈ [a,b]$.
  \pause
  \begin{lstlisting}
    maxNum a b = foldl1 help $ map (\m -> (m, num m)) [a..b]
      where help (m, n) (m', n')
              | n' > n    = (m', n')
              | otherwise = (m, n)
  \end{lstlisting}
  Oder:
  \begin{lstlisting}
    import Data.List -- for maximumBy
    maxNum a b = maximumBy help $ map (\m -> (m, num m)) [a..b]
      where help (m,n) (m',n') = compare n n'
  \end{lstlisting}
  (\lstinline{help} kann man auch als \lstinline{compare `on` snd} schreiben – \lstinline{on}, aus \lstinline{Data.Function}, ist eine Variante des Verkettungsoperators \lstinline{.} für Funktionen mit zwei Argumenten.)
\end{frame}

\begin{frame}[fragile]
  \frametitle{\lstinline{primepowers}}
  Definiere Funktion \lstinline{primepowers}, die für gegebenen Parameter $n$ die unendliche Liste der ersten $n$ Potenzen aller Primzahlen berechnet, aufsteigend sortiert.
  Bereits vorgegeben sind die unendliche, aufsteigend sortierte Liste \lstinline{primes} aller Primzahlen sowie \lstinline{merge} vom ersten Übungsblatt.
  (Der Potenzoperator in Haskell ist \lstinline{^}.)
  \pause
  \begin{lstlisting}
    primepowers 1 = primes
    primepowers n = merge
      (map (^n) primes)
      (primepowers (n-1))
  \end{lstlisting}
  Oder:
  \begin{lstlisting}
    primepowers n = foldr merge []
      [map (^i) primes | i <- [1..n]]
  \end{lstlisting}
  Aber nicht:
  \begin{lstlisting}
    primepowers n = foldr merge []
      [map (p^) [1..n] | p <- primes]
  \end{lstlisting}
\end{frame}

\begin{frame}[fragile]
  \frametitle{\lstinline{intersect}}
  Definiere Funktion \lstinline{intersect}, die den Schnitt zweier (möglicherweise unendlicher) Mengen berechnet.
  Eine Menge wird dargestellt als aufsteigend sortierte Liste ihrer Elemente ohne doppelte Elemente.
  \pause
  \begin{lstlisting}
    intersect (x:xs) (y:ys)
      | x == y = x : intersect xs ys
      | x < y  = intersect xs (y:ys)
      | x > y  = intersect (x:xs) ys
    intersect xs ys = []
  \end{lstlisting}
\end{frame}

\begin{frame}[fragile]
  \frametitle{\lstinline{intersectAll}}
  Definiere Funktion \lstinline{intersectAll}, die für endliche, nichtleere Liste von Mengen den Schnitt aller Mengen berechnet.
  \pause
  \begin{lstlisting}
    intersectAll (l:ls) = foldr intersect l ls
    -- oder
    intersectAll = foldr1 intersect
  \end{lstlisting}
\end{frame}

\begin{frame}[fragile]
  \frametitle{\lstinline{intersectAll []}}
  Viele Abgaben hatten diesen Spezialfall:
  \begin{lstlisting}
    intersectAll [] = []
  \end{lstlisting}
  Mathematisch gesehen wäre aber die Grundmenge sinnvoller, damit gilt:
  \begin{align*}
    & \bigcap \{ M_1, M_2, M_3 \} \\
    = & M_1 \cap \left(\bigcap \{ M_2, M_3 \}\right) \\
    = & M_1 \cap \left(M_2 \cap \left(\bigcap \{ M_3 \}\right)\right) \\
    = & M_1 \cap \left(M_2 \cap \left(M_3 \cap \left(\bigcap \{\}\right)\right)\right)
  \end{align*}
  Die Grundmenge ist aber kein gültiger Wert,
  und um die Invariante \lstinline{intersectAll (x:xs) == intersect x $ intersectAll xs} auch für \lstinline{xs = []} nicht zu verletzen,
  muss \lstinline{intersectAll []} einen Fehler ergeben
  (z.\,B. \lstinline{undefined}).
\end{frame}

\begin{frame}[fragile]
  \frametitle{\lstinline{commonMultiples}}
  Definiere Funktion \lstinline{commonMultiples}, so dass \lstinline{commonMultiples a b c} die Liste aller gemeinsamen Vielfachen von \lstinline{a}, \lstinline{b} und \lstinline{c} bestimmt.
  \pause
  \begin{lstlisting}
    commonMultiples a b c = intersectAll
      [multiples a, multiples b, multiples c]
      where multiples n = map (*n) [1..]
  \end{lstlisting}
\end{frame}

\begin{frame}
  \frametitle{\lstinline{hamming}}
  Definiere unendliche Liste \lstinline{hamming} aller \emph{Hamming}-Zahlen, also aller Zahlen der Gestalt
  \[h = 2^i \cdot 3^j \cdot 5^k \text{ mit } i,j,k ≥ 0\]
  Mögliche Lösungen:
  \begin{itemize}
  \item List Comprehension
  \item \lstinline{merge} mit folgenden Eigenschaften der Hamming-Zahlen:
    \begin{itemize}
    \item 1 ist die erste Hamming-Zahl
    \item alle anderen Hamming-Zahlen sind $2h$, $3h$ oder $5h$ für eine schon berechnete Hamming-Zahl $h$
    \end{itemize}
  \end{itemize}
  (Es war nicht gefordert, dass die Liste sortiert ist oder jede Zahl nur einmal vorkommt.)
\end{frame}

\begin{frame}[fragile]
  \frametitle{\lstinline{hamming} – List Comprehension}
  \begin{lstlisting}
    hamming = [2^i * 3^j * 5^k | i<-[0..], j<-[0..], k<-[0..]]
  \end{lstlisting}
  Nicht korrekt – da der letzte Generator (für \lstinline{k}) nie erschöpft wird,
  werden \lstinline{j} und \lstinline{i} nie einen anderen Wert als \lstinline{0} annehmen,
  die meisten Hamming-Zahlen werden also nie erreicht.
  \pause
  \begin{lstlisting}
    hamming = [2^i * 3^j * 5^k | n <- [0..],
                                 i <- [0..n],
                                 j <- [0..(n-i)],
                                 let k = n-i-j]
  \end{lstlisting}
  Ein unendlicher Generator mit Limit, gegen das man die anderen Generatoren laufen lässt.
\end{frame}

\begin{frame}[fragile]
  \frametitle{\lstinline{hamming} – \lstinline{merge}}
  \begin{lstlisting}
    hamming = 1 : (map (2*) hamming) `merge`
                  (map (3*) hamming) `merge`
                  (map (5*) hamming)
  \end{lstlisting}
  Wenn man \lstinline{merge} so anpasst, dass Duplikate entfernt werden,
  ist diese Lösung sogar sortiert \emph{und} frei von Mehrfachelementen.
\end{frame}

\begin{frame}[fragile]
  \frametitle{Algebraische Datentypen}
  \emph{Algebraische Datentypen} setzen einen Typ aus einem oder mehreren Konstruktoren zusammen.
  % GitHub ghc/ghc repository, tag ghc-8.0.1-release, libraries/base/GHC/Base.hs, L183+L178
  % (Note that this is “for use when compiling GHC.Base itself doesn't work”, hidden with an #if 0;
  % the real Bool is buried in compiler/prelude/TysWiredIn.hs, full of magic,
  % whereas the real () is defined just like this but in libraries/ghc-prim/GHC/Tuple.hs.)
  \begin{lstlisting}
    -- GHC.Base
    data  Bool  =  False | True
    data  ()  =  ()
  \end{lstlisting}
  Konstruktoren können Parameter haben.
  % GitHub ghc/ghc repository, tag ghc-8.0.1-release, libraries/base/GHC/IO/Exception.hs, L224-226
  \begin{lstlisting}
    -- GHC.IO.Exception
    data ExitCode
      = ExitSuccess
      | ExitFailure Int
  \end{lstlisting}
  Zugriff funktioniert über Pattern Matching:
  \begin{lstlisting}
    exitCode ExitSuccess = 0
    exitCode (ExitFailure code) = code
  \end{lstlisting}
\end{frame}

\begin{frame}[fragile]
  \frametitle{Algebraische Datentypen}
  Datentypen können auch Typparameter haben, die dann in den Konstruktoren verwendet werden können.
  % GitHub ghc/ghc repository, tag ghc-8.0.1-release, libraries/base/GHC/Base.hs, L204
  \begin{lstlisting}
    -- GHC.Base
    data  Maybe a  =  Nothing | Just a
  \end{lstlisting}
  % GitHub ghc/ghc repository, tag ghc-8.0.1-release, libraries/base/Data/Either.hs, L124
  \begin{lstlisting}
    -- Data.Either
    data  Either a b  =  Left a | Right b
  \end{lstlisting}
\end{frame}

\end{document}
