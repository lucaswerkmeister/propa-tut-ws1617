\documentclass{beamer}
\usepackage{tut}

\def\tuttitle{Syntaktische Analyse, Codeerzeugung}
\date{2017-02-07}

\begin{document}
\normalsize
\normalem

\lstset{language=jasmin}

\begin{frame}[plain]
  \titlepage
\end{frame}

\begin{frame}
  \frametitle{First-/Follow-Mengen}
  Betrachte folgende Grammatik (Terminale unterstrichen):
  \begin{grammar}
    \nonterminal{S} &→ \nonterminal{L} \terminal{=} \nonterminal{R} \gor \nonterminal{R} \\
    \nonterminal{L} &→ \terminal{*} \nonterminal{R} \gor \terminal{id} \\
    \nonterminal{R} &→ \nonterminal{L}
  \end{grammar}
  Bestimme Mengen \First1 und \Follow1 für alle Nichtterminale.
  Für alle Terminale in den \Follow1-Mengen soll Folge von Ableitungsschritten angegeben werden,
  die belegt, dass das Terminal Teil der \Follow1-Menge ist.
  \pause
  \begin{align*}
    \First1(\nonterminal{L}) &= \{\terminal{*}, \terminal{id}\} & \Follow1(\nonterminal{L}) &= \{\terminal{\#}, \terminal{=}\} \\
    \First1(\nonterminal{R}) &= \First1(\nonterminal{L}) = \{\terminal{*}, \terminal{id}\} & \Follow1(\nonterminal{R}) &= \{\terminal{\#}, \terminal{=}\} \\
    \First1(\nonterminal{S}) &= \First1(\nonterminal{L}) \cup \First1(\nonterminal{R}) & \quad \Follow1(\nonterminal{S}) &= \{\terminal{\#}\} \\
    &= \{\terminal{*}, \terminal{id}\}
  \end{align*}
\end{frame}

\begin{frame}
  \frametitle{Java Bytecode: Stack}
  Die \textit{Java Virtual Machine} ist eine \emph{Stack-Maschine};
  viele Befehle, insbesondere alle arithmetischen Operationen,
  nehmen Operanden vom Stack und schreiben sie wieder dorthin.
  
  Außerdem haben diese Befehle ein typspezifisches Präfix:
  \lstinline{iadd} addiert zwei \lstinline[language=Java]{int}s,
  \lstinline{fmul} multipliziert zwei \lstinline[language=Java]{float}s,
  usw.
  
  Wenn also auf dem Stack bereits die \lstinline[language=Java]{int}-Werte \lstinline{3}, \lstinline{2} und \lstinline{1} liegen,
  liegt nach dem Ausführen der Befehle \lstinline{iadd}, \lstinline{imul} am Ende der Wert \lstinline{9} auf dem Stack.
\end{frame}

\begin{frame}
  \frametitle{Java Bytecode: Frames}
  Zu jeder Methode gehört ein \textit{Frame}:
  der Bereich, in dem alle lokalen Variablen der Methode stehen,
  darunter alle ihre Parameter und, wenn die Methode nicht statisch ist, der \lstinline[language=Java]{this}-Pointer.
  Der Austausch zwischen Frame und Stack geschieht über zwei Befehle
  (jeweils in einer Variante pro Typ):
  \textbf{\texttt{\_load X}} lädt die $X$. lokale Variable auf den Stack,
  \textbf{\texttt{\_store X}} nimmt einen Wert vom Stack und schreibt ihn in die $X$. lokale Variable.
  
  Weil die lokalen Variablen mit den niedrigsten Indizes am häufigsten verwendet werden,
  haben sie eigene Befehle:
  \lstinline{iload\_1} ist gleichbedeutend zu \lstinline{iload 1},
  aber im Bytecode ein Byte kürzer.
  Außerdem gibt es Befehle,
  um häufig verwendete Konstanten zu laden,
  etwa \lstinline{iconst\_0} und \lstinline{aconst\_null}
  (\lstinline{a} steht für den Referenz-Typ).
\end{frame}

\begin{frame}[fragile]
  \frametitle{Codeerzeugung}
  Gegeben sei folgender Ausdruck (in gewohnter Präzedenz): \lstinline{a * (b - 1) + c}.
  Geben Sie ihn in umgekehrter polnischer Notation (\emph{Reverse Polish Notation}, \emph{RPN}) an.
  \pause
  \begin{lstlisting}
    a b 1 - * c +
  \end{lstlisting}
  \pause
  Geben Sie Java-Bytecode an, der den Ausdruck berechnet.
  Die \lstinline[language=Java]{int}-Variablen \lstinline{a}, \lstinline{b} und \lstinline{c} seien als lokale Variablen 1, 2 und 3 verfügbar.
  \pause
  \begin{lstlisting}
    iload_1
    iload_2
    iconst_1
    isub
    imul
    iload_3
    iadd
  \end{lstlisting}
\end{frame}

\begin{frame}
  \frametitle{Java Bytecode: Konstantenpool, Aufrufe, Tests}
  Jede Klasse hat einen Konstantenpool, der verschiedene Konstanten hält:
  Zahlen, Strings, Referenzen auf Felder und Methoden.
  Der Befehl \lstinline{ldc X} lädt die Konstante \texttt{X} aus dem Konstantenpool auf den Stack.
  
  Ein normaler Methodenaufruf wird mit dem Befehl \lstinline{invokevirtual X} ausgeführt.
  Dabei sollten das Zielobjekt und die Argumente auf dem Stack liegen,
  und \texttt{X} ist die Stelle im Konstantenpool,
  an der die Referenz auf die aufzurufende Methode steht.
  
  Für bedingte Sprünge gibt es die \textbf{\texttt{if\_\_ X}}-Befehle,
  die Operanden auf dem Stack vergleichen und gegebenenfalls an Stelle \texttt{X} springen
  (Beispiel: \lstinline{if_icmpgt}, \lstinline{if_icmpeq}, …).
  \lstinline{goto X} führt einen unbedingten Sprung aus.
\end{frame}

\begin{frame}[fragile]
  \frametitle{Codeerzeugung}
  Übersetze folgendes Java-Programm in Java-Bytecode.
  \begin{lstlisting}[language=Java]
    if ((a > c) || (c > 0))
        a = a - this.f(a * (b - 1) + c);
  \end{lstlisting}
  Die Referenz \lstinline[language=Java]{this} und die \lstinline[language=Java]{int}s \lstinline{a}, \lstinline{b} und \lstinline{c} stehen in lokalen Variablen 0, 1, 2, 3;
  die Methode \lstinline{f} steht im Konstantenpool an Stelle 2.
  \pause
  \begin{lstlisting}[language=jasmin,autogobble=false]
    iload_1                   iload_1           
    iload_3                   iload_2           
    if_icmpgt then            iconst_1          
    goto no_shortcut          isub              
no_shortcut:                  imul              
    iload_3                   iload_3           
    iconst_0                  iadd              
    if_icmpgt then            invokevirtual #2  
    goto else                 isub              
then:                         istore_1          
    iload_1                   goto finish       
    aload_0               else:                 
                          finish:
  \end{lstlisting}
\end{frame}

\begin{frame}[fragile]
  \frametitle{Bytecode-Erzeugung mit Haskell}
  \lstset{language=Haskell}
  Datentyp für arithmetische Ausdrücke in Baumdarstellung (AST):
  \begin{lstlisting}
    data Exp = Const Int | Var Int | Neg Exp | Add Exp Exp
  \end{lstlisting}
  Definiere Funktion \lstinline{codegen :: Exp -> [String]}, die Java-Bytecode für den Ausdrucksbaum erzeugt. Beispiel:
  \begin{lstlisting}
    codegen (Neg (Const 42)) = ["iconst 42", "ineg"]
  \end{lstlisting}
  Um minimalen Stackverbrauch zu gewährleisten, soll bei Knoten mit mehreren Unterbäumen der Code für den höchsten Baum zuerst generiert werden;
  dazu ist \lstinline{height :: Exp -> Int} vorgegeben.
  (\lstinline{Int -> String} macht die Funktion \lstinline{show}.)
\end{frame}

\begin{frame}[fragile]
  \frametitle{Bytecode-Erzeugung mit Haskell}
  \lstset{language=Haskell}
  \begin{lstlisting}
    codegen (Const value) = [ "ldc " ++ (show value) ]
    codegen (Var index)   = [ "iload " ++ (show index) ]
    codegen (Neg expr)    = codegen expr ++ [ "ineg" ]
    codegen (Add e1 e2)
      | height e1 >= height e2  = codegen e1 ++
                                  codegen e2 ++
                                  [ "iadd" ]
      | otherwise               = codegen e2 ++
                                  codegen e1 ++
                                  [ "iadd" ]
  \end{lstlisting}
\end{frame}

\begin{frame}
  \frametitle{Bytecode-Erzeugung mit Haskell}
  Anmerkung zum Aufgabentitel:
  in der Klausur kommen gerne Aufgaben dran, die mehrere Themen kombinieren:
  hier Haskell (funktionale Programmierung) und Bytecode (Compiler).
  Oft dient eines der Themen aber nur zur „Motivation“ / „Aufgabenstellung“,
  und die Aufgabe dreht sich hauptsächlich um das andere Thema.
  Daher: nicht eine Aufgabe nur nach dem Titel als „kann ich nicht“ überspringen!
\end{frame}

\end{document}
