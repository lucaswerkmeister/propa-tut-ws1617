\documentclass{beamer}
\usepackage{tut}

\def\tuttitle{Syntaktische Analyse, Codeerzeugung}
\date{2017-02-07}

\begin{document}
\normalsize
\normalem

\lstset{language=Java}

\begin{frame}[plain]
  \titlepage
\end{frame}

\begin{frame}
  \frametitle{First-/Follow-Mengen}
  Betrachte folgende Grammatik (Terminale unterstrichen):
  \begin{grammar}
    \nonterminal{S} &→ \nonterminal{L} \terminal{=} \nonterminal{R} \gor \nonterminal{R} \\
    \nonterminal{L} &→ \terminal{*} \nonterminal{R} \gor \terminal{id} \\
    \nonterminal{R} &→ \nonterminal{L}
  \end{grammar}
  Bestimme Mengen \First1 und \Follow1 für alle Nichtterminale.
  Für alle Terminale in den \Follow1-Mengen soll Folge von Ableitungsschritten angegeben werden,
  die belegt, dass das Terminal Teil der \Follow1-Menge ist.
  \pause
  \begin{align*}
    \First1(\nonterminal{L}) &= \{\terminal{*}, \terminal{id}\} & \Follow1(\nonterminal{L}) &= \{\terminal{\#}, \terminal{=}\} \\
    \First1(\nonterminal{R}) &= \First1(\nonterminal{L}) = \{\terminal{*}, \terminal{id}\} & \Follow1(\nonterminal{R}) &= \{\terminal{\#}, \terminal{=}\} \\
    \First1(\nonterminal{S}) &= \First1(\nonterminal{L}) \cup \First1(\nonterminal{R}) & \quad \Follow1(\nonterminal{S}) &= \{\terminal{\#}\} \\
    &= \{\terminal{*}, \terminal{id}\}
  \end{align*}
\end{frame}

\end{document}
