\documentclass{beamer}
\usepackage{tut}

\def\tuttitle{Typsystem, Lambda-Kalkül}
\date{2016-11-21/22}

\begin{document}
\normalsize
\normalem

\begin{frame}[plain]
  \titlepage
\end{frame}

\begin{frame}
  \frametitle{Typen und Typklassen in Haskell}
  Geben Sie für folgende Haskell-Funktionen die allgemeinstmöglichen Typen einschließlich etwaiger Typklasseneinschränkungen an.
  Begründen Sie, warum diese Einschränkungen nötig sind.
  
  (Der erste Teil geht natürlich mit GHCi + \texttt{:t}; wichtig ist die Erklärung.)
\end{frame}

\begin{frame}[fragile]
  \frametitle{\lstinline{fun1}}
  \begin{lstlisting}
    fun1 xs = (xs == [])
  \end{lstlisting}
  \pause
  \begin{lstlisting}
    :t fun1
    fun1 :: Eq t => [t] -> Bool
  \end{lstlisting}
  \pause
  \lstinline{Eq} ist nötig wegen
  \begin{lstlisting}
    instance Eq a => Eq [a]
  \end{lstlisting}
  Gleichheit von Listen ist nur für Listen von \lstinline{==}-baren Elementen definiert.
  
  \pause
  Eine gleichwertige Definition, die ohne \lstinline{Eq} auskommt:
  \begin{lstlisting}
    fun1 [] = True
    fun1 (x:xs) = False
  \end{lstlisting}
  oder einfach \lstinline{fun1 = null}.
\end{frame}

\begin{frame}[fragile]
  \frametitle{\lstinline{fun2}}
  \begin{lstlisting}
    fun2 f a = foldr f "a"
  \end{lstlisting}
  \pause
  \begin{lstlisting}
    :t fun2
    
    -- GHC <= 7.8:
    fun2 ::
      (a -> [Char] -> [Char]) -> t -> [a] -> [Char]
    
    -- GHC >= 7.10:
    fun2 ::
      Foldable t1 =>
      (a -> [Char] -> [Char]) -> t -> t1 a -> [Char]
  \end{lstlisting}
\end{frame}

\begin{frame}[fragile]
  \frametitle{Exkurs: \lstinline{Foldable}}
  \lstinline{Foldable}: neue Abstraktion in GHC 7.10.
  Definiert
  \lstinline{foldr}, \lstinline{foldl},
  \lstinline{null}, \lstinline{length}, \lstinline{elem},
  \lstinline{minimum}, \lstinline{maximum}, \lstinline{sum}, \lstinline{product}
  und einige andere Operationen.
  Instanzen:
  \begin{lstlisting}
    instance Foldable [] -- Defined in Data.Foldable
    instance Foldable Maybe -- Defined in Data.Foldable
    instance Foldable (Either a) -- Defined in Data.Foldable
    instance Foldable ((,) a) -- Defined in Data.Foldable
  \end{lstlisting}
  Beachte: allen Instanzen fehlt noch ein Typargument.
  Die Signatur
  \begin{lstlisting}
    null :: Foldable t => t a -> Bool
  \end{lstlisting}
  bedeutet dann zum Beispiel für \lstinline{t = []}
  \begin{lstlisting}
    null :: [] a -> Bool
  \end{lstlisting}
  und \lstinline{[] a} bedeutet \lstinline{[a]}.
\end{frame}

\begin{frame}[fragile]
  \frametitle{Exkurs: \lstinline{Foldable}}
  Auch 2-Tupel sind \lstinline{Foldable} (auf dem zweiten Element),
  mit lustigen Ergebnissen:
  \begin{lstlisting}
    Prelude> length (1,2)
    1
    Prelude> elem 1 (1,2)
    False
    Prelude> minimum (1,2)
    2
    Prelude> sum (1,2)
    2
  \end{lstlisting}
\end{frame}

\begin{frame}
  \frametitle{Exkurs: \lstinline{Foldable}}
  In der Klausur gilt:
  wenn eine Lösung in irgendeiner Haskell-Version gültig ist,
  wird sie auch akzeptiert.
  Für \lstinline{fun2} wären also sowohl die Signatur mit Liste als auch die mit \lstinline{Foldable} in Ordnung.
  
  (Disclaimer: das ist eine Folie vom letzten Jahr, dieses Jahr habe ich noch nicht nachgefragt.
  Falls sich was ändert, sage ich euch Bescheid.)
\end{frame}

\begin{frame}[fragile]
  \frametitle{\lstinline{fun3}}
  \begin{lstlisting}
    fun3 f a xs c = foldl f a xs c
    -- der Einfachkeit halber sei foldl ohne Foldable:
    foldl :: (b -> a -> b) -> b -> [a] -> b
  \end{lstlisting}
  \pause
  \begin{lstlisting}
    :t fun3
    fun3 ::
      ((b -> c) -> a -> b -> c) ->
      (b -> c) -> [a] -> b -> c
  \end{lstlisting}
\end{frame}

\begin{frame}[fragile]
  \frametitle{\lstinline{fun4}}
  \begin{lstlisting}
    fun4 f xs = map f xs xs
  \end{lstlisting}
  \pause
  Nicht typisierbar.
  \begin{lstlisting}
    :t map
    map :: (a -> b) -> [a] -> [b]
  \end{lstlisting}
  Typ \lstinline{[b]} von \lstinline{map f xs} ist kein Funktionstyp,
  kann nicht mit \lstinline{xs} aufgerufen werden.
\end{frame}

\begin{frame}[fragile]
  \frametitle{\lstinline{fun5}}
  \begin{lstlisting}
    fun5 a b c = (maximum [a..b], 3 * c)
  \end{lstlisting}
  \pause
  Hilfestellung: Listenschreibweise \lstinline{[a..b]} ist über Typklasse \lstinline{Enum} definiert.
  \pause
  \begin{lstlisting}
    :t fun5
    fun5 ::
      (Enum l, Ord l, Num n) =>
      l -> l -> n -> (l, n)
  \end{lstlisting}
  \begin{itemize}
  \item \lstinline{a} und \lstinline{b} müssen den gleichen Typ \lstinline{l} haben
  \item Für \lstinline{[a..b]} muss \lstinline{Enum l} gelten
  \item Für \lstinline{maximum [a..b]} muss \lstinline{Ord l} gelten
  \item \lstinline{c} hat unabhängigen Typ \lstinline{n}
  \item Für \lstinline{3 * c} muss \lstinline{Num n} gelten
  \end{itemize}
\end{frame}

\begin{frame}[fragile]
  \frametitle{\lstinline{fun6}}
  \begin{lstlisting}
    fun6 x y =
      succ (toEnum (last [fromEnum x..fromEnum y]))
  \end{lstlisting}
  \pause
  \begin{lstlisting}
    :t fun6
    fun6 ::
      (Enum s, Enum t, Enum u) =>
      s -> t -> u
  \end{lstlisting}
  \lstinline{s}, \lstinline{t} und \lstinline{u} können unterschiedliche Typen sein,
  weil sie durch \lstinline{fromEnum} und \lstinline{toEnum} voneinander entkoppelt sind.
  Was \lstinline{u} (Rückgabetyp) ist, wird erst beim Aufruf festgelegt, vergleiche:
  \begin{lstlisting}
    Prelude> (toEnum 65) :: Char
    'A'
    Prelude> (toEnum 65) :: Int
    65
  \end{lstlisting}
\end{frame}

\begin{frame}[fragile]
  \frametitle{\lstinline{fun7}}
  \begin{lstlisting}
    fun7 x = if show x /= [] then x else error
  \end{lstlisting}
  \pause
  Hilfestellung:
  \begin{lstlisting}
    show :: Show a => a -> String
  \end{lstlisting}
  \pause
  Nicht typisierbar ohne Language Extension.
  Alle Verzweigungen müssen den selben Typ zurückgeben,
  und da \lstinline{error :: [Char] -> a} eine Funktion ist,
  muss auch \lstinline{x} eine Funktion sein,
  die aber für \lstinline{show x} gleichzeitig \lstinline{Show} erfüllen muss.
  \begin{lstlisting}
    fun7 ::
      Show ([Char] -> a) =>
      ([Char] -> a) -> [Char] -> a
  \end{lstlisting}
  \begin{quote}
    Non type-variable argument in the constraint: \lstinline{Show ([Char] -> a)} \\
    (Use FlexibleContexts to permit this)
  \end{quote}
\end{frame}

\begin{frame}[fragile]
  \frametitle{Abstrakte Syntaxbäume}
  Definiere Datentyp \lstinline{Exp t} für Ausdrücke.
  Ein Ausdruck ist eine Variable, eine \lstinline{Integer}-Konstante, die Summe zweier Ausdrücke oder das logische Und zweier Ausdrücke (in C-Logik).
  Der Typ von Variablennamen ist nicht spezifiziert, sondern über die Typvariable \lstinline{t} ausgelagert.
  \pause
  \begin{lstlisting}
    data Exp t
      = Var t
      | Const Integer
      | Add (Exp t) (Exp t)
      | And (Exp t) (Exp t)
      deriving Eq
  \end{lstlisting}
  (\lstinline{deriving Eq} brauchen wir später zum Optimieren.)
\end{frame}

\begin{frame}[fragile]
  \frametitle{\lstinline{eval}}
  Gegeben ist ferner der Typ
  \begin{lstlisting}
    type Env t = t -> Integer
  \end{lstlisting}
  der eine \emph{Variablenumgebung} darstellt.
  Definiere damit
  \begin{lstlisting}
    eval :: Env t -> Exp t -> Integer
  \end{lstlisting}
  um einen Ausdruck komplett auszuwerten.
  \pause
  \begin{lstlisting}
    eval env (Var v) = env v
    eval env (Const c) = c
    eval env (Add e1 e2) = eval env e1 + eval env e2
    eval env (And e1 e2) =
      if eval env e1 /= 0 && eval env e2 /= 0
      then 1 else 0
  \end{lstlisting}
\end{frame}

\begin{frame}[fragile]
  \frametitle{\lstinline{Show Exp}}
  Definiere Instanz der Typklasse \lstinline{Show} für \lstinline{Exp},
  um einen Ausdruck in Infix-Notation darzustellen.
  Beispiel: \lstinline{show (Add (Const 1) (Const 2)) = "(1 + 2)"}
  \pause
  \begin{lstlisting}
    instance Show t => Show (Exp t) where
      show (Var v) = show v
      show (Const c) = show c
      show (Add e1 e2) =
        "(" ++ show e1 ++ " + " ++ show e2 ++ ")"
      show (And e1 e2) =
        "(" ++ show e1 ++ " && " ++ show e2 ++ ")"
  \end{lstlisting}
\end{frame}

\begin{frame}[fragile]
  \frametitle{\lstinline{cfold}}
  Definiere
  \begin{lstlisting}
    cfold :: Exp t -> Exp t
  \end{lstlisting}
  um einen Ausdruck mit Konstantenfaltung zu optimieren.
  Beispiel: \lstinline{cfold (Add (Const 1) (Const 2)) = Const 3}
  \pause
  \begin{lstlisting}
    cfold' (Add (Const a) (Const b)) = Const (a+b)
    cfold' (And (Const a) (Const b)) =
      Const (if a /= 0 && b /= 0 then 1 else 0)
    cfold' x = x
    
    cfold (Add e1 e2) = cfold' (Add (cfold e1) (cfold e2))
    cfold (And e1 e2) = cfold' (And (cfold e1) (cfold e2))
    cfold x = x
  \end{lstlisting}
\end{frame}

\begin{frame}[fragile]
  \frametitle{\lstinline{simplify}}
  Definiere
  \begin{lstlisting}
    simplify :: Exp t -> Exp t
  \end{lstlisting}
  um einen Ausdruck unter Ausnutzung algebraischer Identitäten,
  etwa $x+0=0+x=x$ oder $x∧0=0∧x=0$,
  zu optimieren.
  \pause
  \begin{lstlisting}
    simplify' (Add e1 (Const 0)) = e1
    simplify' (Add (Const 0) e2) = e2
    simplify' (And e1 (Const 0)) = Const 0
    simplify' (And (Const 0) e2) = Const 0
    simplify' (And e1 (Const _)) = e1
    simplify' (And (Const _) e2) = e2
    
    simplify (Add e1 e2) =
      simplify' (Add (simplify e1) (simplify e2))
    simplify (And e1 e2) =
      simplify' (And (simplify e1) (simplify e2))
    simplify x = x
  \end{lstlisting}
\end{frame}

\begin{frame}[fragile]
  \frametitle{\lstinline{optimize}}
  Konstantenfaltung und Vereinfachung können gegenseitig voneinander profitieren.
  Für welchen Ausdruck muss man erst Konstanten falten und dann vereinfachen?
  \pause
  \begin{lstlisting}
    And (Var x) (Add (Const 1) (Const (-1)))
    = And (Var x) (Const 0)
    = Const 0
  \end{lstlisting}
  \pause
  Für welchen Ausdruck muss man erst vereinfachen und dann Konstanten falten?
  \pause
  \begin{lstlisting}
    Add ((And (Var x) (Const 0))) (Const 0)
    = Add (Const 0) (Const 0)
    = Const 0
  \end{lstlisting}
\end{frame}

\begin{frame}[fragile]
  \frametitle{\lstinline{optimize}}
  Definiere
  \begin{lstlisting}
    optimize :: Eq t => Exp t -> Exp t
  \end{lstlisting}
  um beide Optimierungen so lange anzuwenden, bis sich der Ausdruck nicht mehr verändert.
  \pause
  \begin{lstlisting}
    optimize x
      | x == ox   = x
      | otherwise = optimize ox
      where ox = cfold $ simplify x
  \end{lstlisting}
  \begin{lstlisting}
    optimize x
      | x /= cfold x    = optimize (cfold x)
      | x /= simplify x = optimize (simplify x)
      | otherwise       = x
  \end{lstlisting}
\end{frame}

\begin{frame}[fragile]
  \frametitle{\lstinline{Tree}}
  Datentyp für binäre Bäume:
  \begin{lstlisting}
    data Tree a = Leaf | Node (Tree a) a (Tree a)
  \end{lstlisting}
  Ein Baum ist \emph{sortiert}, wenn für jeden Knoten \lstinline{Node t1 x t2}
  alle Elemente in \lstinline{t1} $≤$ \lstinline{x} und alle in \lstinline{t2} $≥$ \lstinline{x} sind.
  
  Wir definieren mehrere Operationen, die die Sortierung von Bäumen erhalten sollen,
  allerdings auch für nicht sortierte oder nicht einmal sortierbare Bäume funktionieren sollen:
  wir können also nicht Elemente vergleichen,
  sondern dürfen uns nur auf die Invariante (sortierte Baumstruktur) verlassen.
\end{frame}

\begin{frame}[fragile]
  \frametitle{\lstinline{deleteMin}}
  \begin{lstlisting}
    deleteMin :: Tree a -> (a, Tree a)
  \end{lstlisting}
  löscht aus einem nichtleeren, sortierten Baum das kleinste Element.
  Zurückgegeben wird dieses Element sowie der Baum ohne dieses Element.
  \pause
  \begin{lstlisting}
    deleteMin (Node Leaf x t2) = (x, t2)
    deleteMin (Node t1 x t2) = (y, Node t1' x t2)
      where (y, t1') = deleteMin t1
  \end{lstlisting}
\end{frame}

\begin{frame}[fragile]
  \frametitle{\lstinline{merge}}
  \begin{lstlisting}
    merge :: Tree a -> Tree a -> Tree a
  \end{lstlisting}
  fügt zwei Bäume zusammen.
  Wenn beide Bäume sortiert sind und alle Elemente im ersten $≤$ alle im zweiten sind,
  soll der Ergebnisbaum auch sortiert sein.
  Hinweis: \lstinline{deleteMin} hilft!
  \pause
  \begin{lstlisting}
    merge t1 Leaf = t1
    merge t1 t2 = Node t1 x t2'
      where (x, t2') = deleteMin t2
  \end{lstlisting}
\end{frame}

\begin{frame}[fragile]
  \frametitle{\lstinline{partition}}
  \begin{lstlisting}
    partition :: (a -> Bool) -> Tree a -> (Tree a, Tree a)
  \end{lstlisting}
  teilt einen Baum gemäß einem Prädikat in zwei Teilbäume auf,
  wobei der erste alle Elemente enthält, für die das Prädikat erfüllt ist,
  und der zweite alle anderen.
  \pause
  \begin{lstlisting}
    partition p Leaf = (Leaf, Leaf)
    partition p (Node t1 x t2) = if p x
      then (Node t1t x t2t, merge t1f t2f)
      else (merge t1t t2t, Node t1f x t2f)
      where (t1t, t1f) = partition p t1
            (t2t, t2f) = partition p t2
  \end{lstlisting}
\end{frame}

\end{document}
