\documentclass{beamer}
\usepackage{tut}

\def\tuttitle{C, MPI}
\date{2017-01-23/24}

\begin{document}
\normalsize
\normalem

\lstset{language=C}

\begin{frame}[plain]
  \titlepage
\end{frame}

\begin{frame}
  \frametitle{Nachtrag}
  “song4” vom letzten Mal war übrigens die KIT-Hymne.
\end{frame}

\begin{frame}
  \frametitle{C: Zeiger-Arithmetik, Arrays}
  Welche Ausgabe erzeugt das folgende C-Programm?
  \lstinputlisting[linerange={1-1,3-15}]{tut10/task1.c}
\end{frame}

\begin{frame}[fragile]
  \frametitle{(c)gdb Spickzettel}
  Wichtige Befehle:
  \begin{description}\let\realitem=\item \renewcommand{\item}[1][]{\realitem[\textbf#1]}
  \item[break] Breakpoint setzen
  \item[run] Programm starten
  \item[step] Programm einen Schritt weiterlaufen lassen
  \item[next] Programm einen Schritt weiterlaufen lassen,
    dabei Funktionsaufrufe überspringen
  \item[print] Ausdruck auswerten und Ergebnis anzeigen
  \item[continue] Programm weiterlaufen lassen
  \item[list] Code an der aktuellen Stelle anzeigen lassen
  \end{description}
\end{frame}

\begin{frame}[fragile]
  \frametitle{MPI: Punkt-zu-Punkt-Kommunikation}
  Eine Firma entwickelt folgendes MPI-Programm,
  welches dafür vorgesehen ist, von zwei MPI-Prozessen ausgeführt zu werden.

  Während des Testens durch die Firma verhält sich das Programm wie erwartet.
  Nach der Auslieferung beschweren sich die Kunden jedoch prompt darüber, dass das Programm hängenbleibt.

  \begin{enumerate}
  \item Wieso bleibt das Programm hängen?
  \item Finde möglichst viele Lösungen, das Programm zu verändern, so dass es korrekt funktioniert.
  \end{enumerate}
\end{frame}

\begin{frame}
  \frametitle{MPI: Punkt-zu-Punkt-Kommunikation}
  \lstinputlisting[linerange={6-7,9-22,24-27}]{tut10/task2.c}
\end{frame}

\begin{frame}
  \frametitle{MPI: Punkt-zu-Punkt-Kommunikation – 1}
  Das Programm bleibt hängen, weil beide Prozesse zunächst \lstinline{MPI_Send} und dann \lstinline{MPI_Recv} aufrufen.
  Wenn der \lstinline{MPI_Send}-Aufruf blockiert, bis die Übertragung durch \lstinline{MPI_Recv} am anderen Ende stattfinden kann,
  dann warten beide Prozesse darauf, dass der jeweils andere \lstinline{MPI_Recv} aufruft, ohne dies jedoch selbst zu tun.
  
  Beim Testen in der Firma wurde offenbar eine andere MPI-Implementierung verwendet,
  welche im Standardmodus puffert, d.\,h., \lstinline{MPI_Send} überträgt die Daten in einen Puffer,
  danach geht die Ausführung allerdings weiter, ohne auf \lstinline{MPI_Recv} zu warten.
\end{frame}

\begin{frame}
  \frametitle{MPI: Punkt-zu-Punkt-Kommunikation – 2}
  \begin{itemize}
  \item Kreis aufbrechen:
    Ein Prozess ruft \lstinline{MPI_Recv} vor \lstinline{MPI_Send} auf.
  \item Explizit puffern:
    Puffer mit \lstinline{MPI_Buffer_attach} bereitstellen,
    \lstinline{MPI_Bsend} verwenden.
  \item Asynchron senden:
    \lstinline{MPI_ISend} verwenden.
    Wichtig: anderen Puffer für \lstinline{MPI_Recv} verwenden –
    der Puffer für \lstinline{MPI_Isend} darf nicht verändert werden, bis die Übertragung abgeschlossen ist!
  \item Sende- und Empfangsoperation kombinieren:
    \lstinline{MPI_Sendrecv} oder \lstinline{MPI_Sendrecv_replace} verwenden.
  \item …
  \end{itemize}
\end{frame}

\end{document}
