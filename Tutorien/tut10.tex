\documentclass{beamer}
\usepackage{tut}

\def\tuttitle{C, MPI}
\date{2017-01-23/24}

\begin{document}
\normalsize
\normalem

\lstset{language=C}

\begin{frame}[plain]
  \titlepage
\end{frame}

\begin{frame}
  \frametitle{Nachtrag}
  “song4” vom letzten Mal war übrigens die KIT-Hymne.
\end{frame}

\begin{frame}
  \frametitle{C: Zeiger-Arithmetik, Arrays}
  Welche Ausgabe erzeugt das folgende C-Programm?
  \lstinputlisting[linerange={1-1,3-15}]{tut10/task1.c}
\end{frame}

\begin{frame}[fragile]
  \frametitle{(c)gdb Spickzettel}
  Wichtige Befehle:
  \begin{description}\let\realitem=\item \renewcommand{\item}[1][]{\realitem[\textbf#1]}
  \item[break] Breakpoint setzen
  \item[run] Programm starten
  \item[step] Programm einen Schritt weiterlaufen lassen
  \item[next] Programm einen Schritt weiterlaufen lassen,
    dabei Funktionsaufrufe überspringen
  \item[print] Ausdruck auswerten und Ergebnis anzeigen
  \item[continue] Programm weiterlaufen lassen
  \item[list] Code an der aktuellen Stelle anzeigen lassen
  \end{description}
\end{frame}

\end{document}
