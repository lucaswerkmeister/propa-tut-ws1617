\documentclass{beamer}
\usepackage{tut}

\def\tuttitle{Rekursion, Typprüfung, Prolog}
\date{2016-12-05/06}

\begin{document}
\normalsize
\normalem

\begin{frame}[plain]
  \titlepage
\end{frame}

\begin{frame}
  \frametitle{Turing-Kombinator}
  Turing fand folgenden Fixpunktkombinator:
  \[Θ = (λ?x.~λ?y.~?y~(?x~?x~?y))~(λ?x.~λ?y.~?y~(?x~?x~?y))\]
  Zeige per $β$-Reduktion, dass gilt:
  \[Θ~F \Rightarrow^\ast F~(Θ~F)\]
\end{frame}

\begin{frame}
  \frametitle{Turing-Kombinator}
  Zur Abkürzung sei
  \begin{align*}
    Θ_0 &:= (λ?x.~λ?y.~?y~(?x~?x~?y))
    \intertext{Damit ist also}
    Θ &= Θ_0~Θ_0
    \intertext{und damit}
    Θ~F &= \onslide<2->{Θ_0~Θ_0~F} \\
    \onslide<3->{&= (λ?x.~λ?y.~?y~(?x~?x~?y))~Θ_0~F} \\
    \onslide<4->{&\Rightarrow (λ?y.~?y~(Θ_0~Θ_0~?y))~F} \\
    \onslide<5->{&\Rightarrow F~(Θ_0~Θ_0~F)} \\
    \onslide<6->{&= F~(Θ~F)}
  \end{align*}
\end{frame}

\begin{frame}
  \frametitle{Turing-Kombinator}
  Der Turing-Kombinator ist auch ein allgemeineres Rezept, um eine rekursive Gleichung zu erfüllen.
  Wollen wir etwa einen Kombinator für
  \begin{align*}
    Ξ~F &\Rightarrow^* F~(λ?g.~F~(Ξ~F~(F~Ξ)~?g))
    \intertext{so definieren wir}
    Ξ &= Ξ_0~Ξ_0
    \intertext{und definieren dann $Ξ_0$, indem wir $x$ ($Ξ_0$) und $y$ ($F$) als Parameter annehmen und dann die Rekursionsgleichung abschreiben, mit $?x~?x$ statt $Ξ$ und $?y$ statt $F$:}
    Ξ_0 &:= λ?x~λ?y.~?y~(λ?g.~?y~(?x~?x~?y~(?y~(?x~?x))~?g))
  \end{align*}
\end{frame}

\end{document}
