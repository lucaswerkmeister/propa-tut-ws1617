\documentclass{beamer}
\usepackage{tut}

\def\tuttitle{Rekursion, Typprüfung, Prolog}
\date{2016-12-05/06}

\begin{document}
\normalsize
\normalem

\begin{frame}[plain]
  \titlepage
\end{frame}

\begin{frame}
  \frametitle{Turing-Kombinator}
  Turing fand folgenden Fixpunktkombinator:
  \[Θ = (λ?x.~λ?y.~?y~(?x~?x~?y))~(λ?x.~λ?y.~?y~(?x~?x~?y))\]
  Zeige per $β$-Reduktion, dass gilt:
  \[Θ~F \Rightarrow^\ast F~(Θ~F)\]
\end{frame}

\begin{frame}
  \frametitle{Turing-Kombinator}
  Zur Abkürzung sei
  \begin{align*}
    Θ_0 &:= (λ?x.~λ?y.~?y~(?x~?x~?y))
    \intertext{Damit ist also}
    Θ &= Θ_0~Θ_0
    \intertext{und damit}
    Θ~F &= \onslide<2->{Θ_0~Θ_0~F} \\
    \onslide<3->{&= (λ?x.~λ?y.~?y~(?x~?x~?y))~Θ_0~F} \\
    \onslide<4->{&\Rightarrow (λ?y.~?y~(Θ_0~Θ_0~?y))~F} \\
    \onslide<5->{&\Rightarrow F~(Θ_0~Θ_0~F)} \\
    \onslide<6->{&= F~(Θ~F)}
  \end{align*}
\end{frame}

\begin{frame}
  \frametitle{Turing-Kombinator}
  Der Turing-Kombinator ist auch ein allgemeineres Rezept, um eine rekursive Gleichung zu erfüllen.
  Wollen wir etwa einen Kombinator für
  \begin{align*}
    Ξ~F &\Rightarrow^* F~(λ?g.~F~(Ξ~F~(F~Ξ)~?g))
    \intertext{so definieren wir}
    Ξ &= Ξ_0~Ξ_0
    \intertext{und definieren dann $Ξ_0$, indem wir $x$ ($Ξ_0$) und $y$ ($F$) als Parameter annehmen und dann die Rekursionsgleichung abschreiben, mit $?x~?x$ statt $Ξ$ und $?y$ statt $F$:}
    Ξ_0 &:= λ?x~λ?y.~?y~(λ?g.~?y~(?x~?x~?y~(?y~(?x~?x))~?g))
  \end{align*}
\end{frame}

\begin{frame}
  \frametitle{\vl{is\_zero}}
  Definiere Funktion \vl{is\_zero}, die prüft, ob übergebene Church-Zahl 0 ist.
  Rückgabe soll ein Church-Boolean sein, also $\ctrue$ oder $\cfalse$.
  
  \pause
  Erinnerung: eine Church-Zahl wird mit einer Nachfolgerfunktion und einem Startelement aufgerufen
  und ruft dann die Nachfolgerfunktion $n$~mal auf dem Startelement auf.
  Eine allgemeine Operation auf einer Church-Zahl hat also die Form
  \[λ?n.~?n~s~z\]
  für bestimmte $s$ und $z$.
  
  \pause
  $c_0$ gibt direkt das Startelement zurück,
  das sollte bei uns also $\ctrue$ sein.
  Wenn die Nachfolgerfunktion jemals aufgerufen wird,
  dann war die Zahl nicht 0,
  also geben wir in diesem Fall immer $\cfalse$ zurück.
  \[\vl{is\_zero} = λ?n.~?n~\underbrace{(λ?x.~\cfalse)}_{s}~\underbrace{\ctrue}_{z}\]
\end{frame}

\begin{frame}
  \frametitle{\vl{less\_eq}}
  Definiere Vergleichsfunktion \vl{less\_eq}.
  (Hier und im Rest der Aufgabe dürfen alle bisher definierten Funktionen verwendet werden.)
  \pause
  \[\vl{less\_eq} = λ?m.~λ?n.~\vl{is\_zero}~(\vl{sub}~?m~?n)\]
  (Es gibt keine negativen Church-Zahlen, daher ist z.\,B. $\vl{sub}~c_3~c_5 \Rightarrow^\ast c_0$.)
\end{frame}

\begin{frame}[fragile]
  \frametitle{\vl{fib}}
  Übersetze folgende Haskell-Funktion in $λ$-Kalkül:
  \begin{lstlisting}
    fib :: Integer -> Integer
    fib 0 = 1
    fib 1 = 1
    fib n = fib (n - 1) + fib (n - 2)
  \end{lstlisting}
\end{frame}

\begin{frame}[fragile]
  \frametitle{\vl{fib}}
  \begin{lstlisting}
    fib :: Integer -> Integer
    fib 0 = 1
    fib 1 = 1
    fib n = fib (n - 1) + fib (n - 2)
  \end{lstlisting}
  Erster Schritt: \lstinline{fib} ohne Pattern Matching.
  \pause
  \begin{lstlisting}
    fib n = if n == 0 || n == 1
            then 1
            else fib (n - 1) + fib (n - 2)
  \end{lstlisting}
\end{frame}

\begin{frame}[fragile]
  \frametitle{\vl{fib}}
  \begin{lstlisting}
    fib n = if n == 0 || n == 1
            then 1
            else fib (n - 1) + fib (n - 2)
  \end{lstlisting}
  Zweiter Schritt: zugehöriges Funktional (also ohne Rekursion) in Haskell.
  \pause
  \begin{lstlisting}
    Fib = \fib -> \n -> if n == 0 || n == 1
                        then 1
                        else fib (n - 1) + fib (n - 2)
  \end{lstlisting}
\end{frame}

\begin{frame}[fragile]
  \frametitle{\vl{fib}}
  \begin{lstlisting}
    Fib = \fib -> \n -> if n == 0 || n == 1
                        then 1
                        else fib (n - 1) + fib (n - 2)
  \end{lstlisting}
  Dritter Schritt: \lstinline{Fib} im Lambda-Kalkül.
  \pause
  \begin{align*}
    \vl{Fib} = λ\vr{fib}.~λ\vr{n}.~&(\vl{is\_zero}~(\vl{pred}~\vr{n}))\\
    &c_1\\
    &(\vl{plus}~(\vr{fib}~(\vl{sub}~\vr{n}~c_1))~(\vr{fib}~(\vl{sub}~\vr{n}~c_2)))
  \end{align*}
\end{frame}

\begin{frame}
  \frametitle{\vl{fib}}
  \begin{align*}
    \vl{Fib} = λ\vr{fib}.~λ?n.~&(\vl{is\_zero}~(\vl{pred}~?n))\\
    &c_1\\
    &(\vl{plus}~(\vr{fib}~(\vl{sub}~?n~c_1))~(\vr{fib}~(\vl{sub}~?n~c_2)))
  \end{align*}
  Vierter Schritt: \vl{fib} definieren.
  \pause
  \[\vl{fib} = Y~\vl{Fib}\]
\end{frame}

\begin{frame}[fragile]
  \frametitle{\vl{foo}}
  Übersetze folgende Haskell-Funktion in $λ$-Kalkül:
  \begin{lstlisting}
    foo :: Integer -> Integer
    foo n
      | n <= 100  = foo (foo (n + 11))
      | otherwise = n - 10
  \end{lstlisting}
\end{frame}

\begin{frame}[fragile]
  \frametitle{\vl{foo}}
  \begin{lstlisting}
    foo n = if n <= 100
            then foo (foo (n + 11))
            else n - 10
  \end{lstlisting}
  Funktional:
  \pause
  \begin{lstlisting}
    Foo = \foo -> \n -> if n <= 100
                        then foo (foo (n + 11))
                        else n - 10
  \end{lstlisting}
  Funktional im Lambda-Kalkül:
  \pause
  \begin{align*}
    \vl{Foo} = λ\vr{foo}.~λ\vr{n}.~&(\vl{less\_eq}~\vr{n}~c_{100})\\
    &(\vr{foo}~(\vr{foo}~(\vl{plus}~\vr{n}~c_{11})))\\
    &(\vl{sub}~\vr{n}~c_{10})
  \end{align*}
  \vl{foo} im Lambda-Kalkül:
  \pause
  \[\vl{foo} = Y~\vl{Foo}\]
\end{frame}

\begin{frame}[fragile]
  \frametitle{\vl{foo}}
  Laut Beispiellösung ist \lstinline{foo} übrigens „die bekannte McCarthy 91-Funktion“,
  welche laut Wikipedia ein Standard-Testfall für formale Beweissysteme ist.
  Ein Beweissystem soll dabei herausfinden, dass diese Funktion die folgende Eigenschaft besitzt:
  \[\vl{foo}(n) = \begin{cases}91 & n \leq 100 \\ n - 10 & n > 100\end{cases}\]
  Daher ist \lstinline{foo} „bekanntlich“ äquivalent zu folgender Definition:
  \begin{lstlisting}
    foo n = if n <= 100 then 91 else n - 10
  \end{lstlisting}
  Damit wäre auch folgende Übersetzung denkbar:
  \[\vl{foo} = λ\vr{n}.~(\vl{less\_eq}~\vr{n}~c_{100})~c_{91}~(\vl{sub}~\vr{n}~c_{10})\]
\end{frame}

\begin{frame}[fragile]
  \frametitle{Rekursionsoperator in Haskell}
  $Y$ ist in Haskell nicht typisierbar.
  Es lässt sich aber ein anderer Rekursionsoperator definieren:
  \begin{lstlisting}
    fix :: (t -> t) -> t
    fix f = f (fix f)
  \end{lstlisting}
  (Findet sich so auch im Modul \lstinline{Data.Function}.)
  
  Definiere damit
  \begin{lstlisting}
    fibF :: (Integer -> Integer) -> (Integer -> Integer)
    fooF :: (Integer -> Integer) -> (Integer -> Integer)
  \end{lstlisting}
  sodass \lstinline{fix fibF}, \lstinline{fix fooF} die gewünschten Funktionen ergeben.
\end{frame}

\begin{frame}[fragile]
  \frametitle{\lstinline{fibF}}
  \begin{lstlisting}
    fib      :: Integer -> Integer
    fib      0 = 1
    fib      1 = 1
    fib      n = fib (n - 1) + fib (n - 2)
  \end{lstlisting}
  \pause
  \begin{lstlisting}
    fibF :: (Integer -> Integer) -> (Integer -> Integer)
    fibF _   0 = 1
    fibF _   1 = 1
    fibF fib n = fib (n - 1) + fib (n - 2)
  \end{lstlisting}
\end{frame}

\begin{frame}[fragile]
  \frametitle{\lstinline{fooF}}
  \begin{lstlisting}
    foo :: Integer -> Integer
    foo     n
      | n <= 100  = foo (foo (n + 11))
      | otherwise = n - 10
  \end{lstlisting}
  \pause
  \begin{lstlisting}
    fooF :: (Integer -> Integer) -> (Integer -> Integer)
    fooF foo n
      | n <= 100  = foo (foo (n + 11))
      | otherwise = n - 10
  \end{lstlisting}
\end{frame}

\end{document}
