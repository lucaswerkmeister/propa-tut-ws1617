\documentclass{beamer}
\usepackage{tut}

\def\tuttitle{Rekursion, Typprüfung, Prolog}
\date{2016-12-05/06}

\begin{document}
\normalsize
\normalem

\begin{frame}[plain]
  \titlepage
\end{frame}

\begin{frame}
  \frametitle{Turing-Kombinator}
  Turing fand folgenden Fixpunktkombinator:
  \[Θ = (λ?x.~λ?y.~?y~(?x~?x~?y))~(λ?x.~λ?y.~?y~(?x~?x~?y))\]
  Zeige per $β$-Reduktion, dass gilt:
  \[Θ~F \Rightarrow^\ast F~(Θ~F)\]
\end{frame}

\begin{frame}
  \frametitle{Turing-Kombinator}
  Zur Abkürzung sei
  \begin{align*}
    Θ_0 &:= (λ?x.~λ?y.~?y~(?x~?x~?y))
    \intertext{Damit ist also}
    Θ &= Θ_0~Θ_0
    \intertext{und damit}
    Θ~F &= \onslide<2->{Θ_0~Θ_0~F} \\
    \onslide<3->{&= (λ?x.~λ?y.~?y~(?x~?x~?y))~Θ_0~F} \\
    \onslide<4->{&\Rightarrow (λ?y.~?y~(Θ_0~Θ_0~?y))~F} \\
    \onslide<5->{&\Rightarrow F~(Θ_0~Θ_0~F)} \\
    \onslide<6->{&= F~(Θ~F)}
  \end{align*}
\end{frame}

\begin{frame}
  \frametitle{Turing-Kombinator}
  Der Turing-Kombinator ist auch ein allgemeineres Rezept, um eine rekursive Gleichung zu erfüllen.
  Wollen wir etwa einen Kombinator für
  \begin{align*}
    Ξ~F &\Rightarrow^* F~(λ?g.~F~(Ξ~F~(F~Ξ)~?g))
    \intertext{so definieren wir}
    Ξ &= Ξ_0~Ξ_0
    \intertext{und definieren dann $Ξ_0$, indem wir $x$ ($Ξ_0$) und $y$ ($F$) als Parameter annehmen und dann die Rekursionsgleichung abschreiben, mit $?x~?x$ statt $Ξ$ und $?y$ statt $F$:}
    Ξ_0 &:= λ?x~λ?y.~?y~(λ?g.~?y~(?x~?x~?y~(?y~(?x~?x))~?g))
  \end{align*}
\end{frame}

\begin{frame}
  \frametitle{\vl{is\_zero}}
  Definiere Funktion \vl{is\_zero}, die prüft, ob übergebene Church-Zahl 0 ist.
  Rückgabe soll ein Church-Boolean sein, also $\ctrue$ oder $\cfalse$.
  
  \pause
  Erinnerung: eine Church-Zahl wird mit einer Nachfolgerfunktion und einem Startelement aufgerufen
  und ruft dann die Nachfolgerfunktion $n$~mal auf dem Startelement auf.
  Eine allgemeine Operation auf einer Church-Zahl hat also die Form
  \[λ?n.~?n~s~z\]
  für bestimmte $s$ und $z$.
  
  \pause
  $c_0$ gibt direkt das Startelement zurück,
  das sollte bei uns also $\ctrue$ sein.
  Wenn die Nachfolgerfunktion jemals aufgerufen wird,
  dann war die Zahl nicht 0,
  also geben wir in diesem Fall immer $\cfalse$ zurück.
  \[\vl{is\_zero} = λ?n.~?n~\underbrace{(λ?x.~\cfalse)}_{s}~\underbrace{\ctrue}_{z}\]
\end{frame}

\begin{frame}
  \frametitle{\vl{less\_eq}}
  Definiere Vergleichsfunktion \vl{less\_eq}.
  (Hier und im Rest der Aufgabe dürfen alle bisher definierten Funktionen verwendet werden.)
  \pause
  \[\vl{less\_eq} = λ?m.~λ?n.~\vl{is\_zero}~(\vl{sub}~?m~?n)\]
  (Es gibt keine negativen Church-Zahlen, daher ist z.\,B. $\vl{sub}~c_3~c_5 \Rightarrow^\ast c_0$.)
\end{frame}

\begin{frame}[fragile]
  \frametitle{\vl{fib}}
  Übersetze folgende Haskell-Funktion in $λ$-Kalkül:
  \begin{lstlisting}
    fib :: Integer -> Integer
    fib 0 = 1
    fib 1 = 1
    fib n = fib (n - 1) + fib (n - 2)
  \end{lstlisting}
\end{frame}

\begin{frame}[fragile]
  \frametitle{\vl{fib}}
  \begin{lstlisting}
    fib :: Integer -> Integer
    fib 0 = 1
    fib 1 = 1
    fib n = fib (n - 1) + fib (n - 2)
  \end{lstlisting}
  Erster Schritt: \lstinline{fib} ohne Pattern Matching.
  \pause
  \begin{lstlisting}
    fib n = if n == 0 || n == 1
            then 1
            else fib (n - 1) + fib (n - 2)
  \end{lstlisting}
\end{frame}

\begin{frame}[fragile]
  \frametitle{\vl{fib}}
  \begin{lstlisting}
    fib n = if n == 0 || n == 1
            then 1
            else fib (n - 1) + fib (n - 2)
  \end{lstlisting}
  Zweiter Schritt: zugehöriges Funktional (also ohne Rekursion) in Haskell.
  \pause
  \begin{lstlisting}
    Fib = \fib -> \n -> if n == 0 || n == 1
                        then 1
                        else fib (n - 1) + fib (n - 2)
  \end{lstlisting}
\end{frame}

\begin{frame}[fragile]
  \frametitle{\vl{fib}}
  \begin{lstlisting}
    Fib = \fib -> \n -> if n == 0 || n == 1
                        then 1
                        else fib (n - 1) + fib (n - 2)
  \end{lstlisting}
  Dritter Schritt: \lstinline{Fib} im Lambda-Kalkül.
  \pause
  \begin{align*}
    \vl{Fib} = λ\vr{fib}.~λ\vr{n}.~&(\vl{is\_zero}~(\vl{pred}~\vr{n}))\\
    &c_1\\
    &(\vl{plus}~(\vr{fib}~(\vl{sub}~\vr{n}~c_1))~(\vr{fib}~(\vl{sub}~\vr{n}~c_2)))
  \end{align*}
\end{frame}

\begin{frame}
  \frametitle{\vl{fib}}
  \begin{align*}
    \vl{Fib} = λ\vr{fib}.~λ?n.~&(\vl{is\_zero}~(\vl{pred}~?n))\\
    &c_1\\
    &(\vl{plus}~(\vr{fib}~(\vl{sub}~?n~c_1))~(\vr{fib}~(\vl{sub}~?n~c_2)))
  \end{align*}
  Vierter Schritt: \vl{fib} definieren.
  \pause
  \[\vl{fib} = Y~\vl{Fib}\]
\end{frame}

\begin{frame}[fragile]
  \frametitle{\vl{foo}}
  Übersetze folgende Haskell-Funktion in $λ$-Kalkül:
  \begin{lstlisting}
    foo :: Integer -> Integer
    foo n
      | n <= 100  = foo (foo (n + 11))
      | otherwise = n - 10
  \end{lstlisting}
\end{frame}

\begin{frame}[fragile]
  \frametitle{\vl{foo}}
  \begin{lstlisting}
    foo n = if n <= 100
            then foo (foo (n + 11))
            else n - 10
  \end{lstlisting}
  Funktional:
  \pause
  \begin{lstlisting}
    Foo = \foo -> \n -> if n <= 100
                        then foo (foo (n + 11))
                        else n - 10
  \end{lstlisting}
  Funktional im Lambda-Kalkül:
  \pause
  \begin{align*}
    \vl{Foo} = λ\vr{foo}.~λ\vr{n}.~&(\vl{less\_eq}~\vr{n}~c_{100})\\
    &(\vr{foo}~(\vr{foo}~(\vl{plus}~\vr{n}~c_{11})))\\
    &(\vl{sub}~\vr{n}~c_{10})
  \end{align*}
  \vl{foo} im Lambda-Kalkül:
  \pause
  \[\vl{foo} = Y~\vl{Foo}\]
\end{frame}

\begin{frame}[fragile]
  \frametitle{\vl{foo}}
  Laut Beispiellösung ist \lstinline{foo} übrigens „die bekannte McCarthy 91-Funktion“,
  welche laut Wikipedia ein Standard-Testfall für formale Beweissysteme ist.
  Ein Beweissystem soll dabei herausfinden, dass diese Funktion die folgende Eigenschaft besitzt:
  \[\vl{foo}(n) = \begin{cases}91 & n \leq 100 \\ n - 10 & n > 100\end{cases}\]
  Daher ist \lstinline{foo} „bekanntlich“ äquivalent zu folgender Definition:
  \begin{lstlisting}
    foo n = if n <= 100 then 91 else n - 10
  \end{lstlisting}
  Damit wäre auch folgende Übersetzung denkbar:
  \[\vl{foo} = λ\vr{n}.~(\vl{less\_eq}~\vr{n}~c_{100})~c_{91}~(\vl{sub}~\vr{n}~c_{10})\]
\end{frame}

\begin{frame}[fragile]
  \frametitle{Rekursionsoperator in Haskell}
  $Y$ ist in Haskell nicht typisierbar.
  Es lässt sich aber ein anderer Rekursionsoperator definieren:
  \begin{lstlisting}
    fix :: (t -> t) -> t
    fix f = f (fix f)
  \end{lstlisting}
  (Findet sich so auch im Modul \lstinline{Data.Function}.)
  
  Definiere damit
  \begin{lstlisting}
    fibF :: (Integer -> Integer) -> (Integer -> Integer)
    fooF :: (Integer -> Integer) -> (Integer -> Integer)
  \end{lstlisting}
  sodass \lstinline{fix fibF}, \lstinline{fix fooF} die gewünschten Funktionen ergeben.
\end{frame}

\begin{frame}[fragile]
  \frametitle{\lstinline{fibF}}
  \begin{lstlisting}
    fib      :: Integer -> Integer
    fib      0 = 1
    fib      1 = 1
    fib      n = fib (n - 1) + fib (n - 2)
  \end{lstlisting}
  \pause
  \begin{lstlisting}
    fibF :: (Integer -> Integer) -> (Integer -> Integer)
    fibF _   0 = 1
    fibF _   1 = 1
    fibF fib n = fib (n - 1) + fib (n - 2)
  \end{lstlisting}
\end{frame}

\begin{frame}[fragile]
  \frametitle{\lstinline{fooF}}
  \begin{lstlisting}
    foo :: Integer -> Integer
    foo     n
      | n <= 100  = foo (foo (n + 11))
      | otherwise = n - 10
  \end{lstlisting}
  \pause
  \begin{lstlisting}
    fooF :: (Integer -> Integer) -> (Integer -> Integer)
    fooF foo n
      | n <= 100  = foo (foo (n + 11))
      | otherwise = n - 10
  \end{lstlisting}
\end{frame}

\begin{frame}
  \frametitle{Let-Polymorphismus}
  Wir wollen den folgenden Ausdruck typisieren:
  \[λ?g.~?g~(f~1)~(f~\xtrue)\]
  Dabei sei $f=λ?x.~?x$.
  
  \pause
  Intuitiv sehen wir: $f$ ist die Identitätsfunktion,
  also hat $f~1$ den Typ $\Tint$ und $f~\xtrue$ den Typ $\Tbool$.
  Damit sollte der ganze Ausdruck den Typ $(\Tint → \Tbool → α) → α$ haben.
  
  \pause
  Was ist hierbei aber der Typ von $f$?\ 
  Erster Ansatz:
  \[(λ?f.~λ?g.~?g~(?f~1)~(?f~\xtrue))~(λ?x.~?x)\]
  Dabei hat in $λ?g.~?g~(?f~1)~(?f~\xtrue)$ $?f$ den Typ $α → α$.
  Damit ist der Ausdruck aber nicht typisierbar, denn $α$ kann nicht gleichzeitig $\Tint$ und $\Tbool$ sein!
\end{frame}

\begin{frame}
  \frametitle{Let-Polymorphismus}
  Stattdessen:
  \[\Let ?f = λ?x.~?x \In λ?g.~?g~(?f~1)~(?f~\xtrue)\]
  Jetzt hat $?f$ innerhalb von $λ?g.~?g~(?f~1)~(?f~\xtrue)$ den Typ:
  \[∀α. α → α\]
  \pause
  Unterschied zu $α → α$:
  Ohne $∀$ ist $α$ eine von außen vorgegebene Typvariable,
  die für \emph{einen} Typ steht.
  Wir garantieren der Umgebung, dass wir für beliebige Belegung von $α$ funktionieren.
  Mit $∀$ dürfen wir $α$ \emph{selbst} frei wählen, beliebig oft,
  und die Umgebung garantiert \emph{uns},
  dass die Funktion vom Typ $∀ α. α → α$ für beliebige (von uns gewählte) Belegung von $α$ funktioniert.
\end{frame}

\begin{frame}[fragile]
  \frametitle{Let-Polymorphismus}
  Das gleiche gibt es übrigens auch in Haskell.
  Normalerweise ist diese Funktion nicht typisierbar:
  \begin{lstlisting}
    x f g = g (f 1) (f True)
  \end{lstlisting}
  Mit einer Language Extension wie \emph{Rank2Types} oder \emph{RankNTypes} können wir allerdings einen gültigen Typ angeben:
  \begin{lstlisting}
    x :: (forall a. a -> a) -> (Int -> Bool -> a) -> a
  \end{lstlisting}
  Die Funktion kann dann zum Beispiel mit \lstinline{id} aufgerufen werden:
  die freie Typvariable in \lstinline{a -> a} wird implizit allquantifiziert
  (\lstinline{id} hat implizit den Typ \lstinline{forall a. a -> a}).
  Im Lambda-Kalkül geschieht das nicht, dort muss man explizit \lstinline{let} verwenden.
\end{frame}

\begin{frame}
  \frametitle{Let-Polymorphismus}
  Änderungen an den Typisierungsregeln:
  \begin{description}
  \item[Var] erlaubt Instanziierungen. Zum Beispiel kann aus $Γ(x) = ∀ α. α → α$ der Typ $Γ \vdash x : \Tint → \Tint$ instanziiert werden, da $∀ α. α → α \succeq \Tint → \Tint$.
  \item[Abs] fordert, dass der Parametertyp kein Typschema ist.
  \item[Let] neue Regel:
    \begin{prooftree}
      \AxiomC{$Γ \vdash t_1 : τ_1$}
      \AxiomC{$Γ, ?x : \mathit{ta}(τ_1, Γ) \vdash t_2 : τ_2$}
      \rulename{Let}
      \BinaryInfC{$Γ \vdash \Let ?x = t_1 \In t_2 : τ_2$}
    \end{prooftree}
    Dabei hat $t_1$ einen normalen (nicht polymorphen) Typ, also mit freien Typvariablen, z.\,B. $α → β → α$.
    $t_2$ darf die polymorphe Version davon nutzen, hier z.\,B. $∀ α. ∀ β. α → β → α$.
    Das \lstinline{let}-Konstrukt schafft den Übergang zwischen den beiden Ausdrücken.
  \end{description}
\end{frame}

\begin{frame}
  \frametitle{Typprüfung}
  Gegeben
  \[Γ = ?a: \Tint, ?b: \Tbool, ?c: \Tchar\]
  soll unter Verwendung der folgenden Instanziierungen
  \begin{align*}
    & ∀α. ∀β. α → β → α \succeq \Tint → \Tbool → \Tint \\
    & ∀α. ∀β. α → β → α \succeq \Tbool → \Tchar → \Tbool
  \end{align*}
  der folgende Herleitungsbaum vervollständigt werden:
  
  \begin{prooftree}
    \small
    \AxiomC{$Γ \vdash λ?x.~λ?y.~?x : α → β → α$}
    \AxiomC{$Γ, ?k : ∀α. ∀β. α → β → α \vdash ?k~?a~(?k~?b~?c) : \Tint$}
    \rulename{Let}
    \BinaryInfC{$Γ \vdash \Let ?k = λ?x.~λ?y.~?x \In ?k~?a~(?k~?b~?c) : \Tint$}
  \end{prooftree}
\end{frame}

\begin{frame}
  \frametitle{Typprüfung}
  Wir zeigen zunächst die linke Seite
  \[Γ \vdash λ?x.~λ?y.~?x : α → β → α\]
  durch mehrmalige Anwendung der Regel
  \pause
  \begin{prooftree}
    \AxiomC{$Γ, ?x: τ_1 \vdash t: τ_2$}
    \rulename{Abs}
    \UnaryInfC{$Γ \vdash λ?x.~t: τ_1 → τ_2$}
  \end{prooftree}
\end{frame}

\begin{frame}
  \frametitle{Typprüfung}
  \begin{prooftree}
    \def\fCenter{(?x}
    \Axiom$(Γ, ?x: α, ?y: β)\fCenter) = α$
    \def\fCenter{\vdash}
    \rulename{Var}
    \UnaryInf$Γ, ?x:α, ?y:β \fCenter ?x: α$
    \rulename{Abs}
    \UnaryInf$Γ, ?x: α \fCenter λ?y.~?x : β → α$
    \rulename{Abs}
    \UnaryInf$Γ \fCenter λ?x.~λ?y.~?x : α → β → α$
  \end{prooftree}
\end{frame}

\begin{frame}
  \frametitle{Typprüfung}
  Jetzt zeigen wir die rechte Seite
  \[Γ, ?k : \underbrace{∀α. ∀β. α → β → α}_{ξ} \vdash ?k~?a~(?k~?b~?c) : \Tint\]
  mit der Regel
  \pause
  \begin{prooftree}
    \AxiomC{$Γ \vdash t_1: τ_2 → τ$}
    \AxiomC{$Γ \vdash t_2: τ_2$}
    \rulename{App}
    \BinaryInfC{$Γ \vdash t_1~t_2: τ$}
  \end{prooftree}
  also
  \begin{prooftree}
    \AxiomC{$Γ, ?k:ξ \vdash ?k~?a: \Tbool → \Tint$}
    \AxiomC{$Γ, ?k:ξ \vdash ?k~?b~?c: \Tbool$}
    \rulename{App}
    \BinaryInfC{$Γ, ?k:ξ \vdash ?k~?a~(?k~?b~?c) : \Tint$}
  \end{prooftree}
\end{frame}

\begin{frame}
  \frametitle{Typprüfung}
  $?k~?a$ ist auch eine Applikation, also gleiche Regel nochmal:
  \begin{prooftree}
    \AxiomC{$Γ, ?k:ξ \vdash ?k: \Tint → \Tbool → \Tint$}
    \AxiomC{$(Γ, ?k:ξ)(?a) = \Tint$}
    \rulename{Var}
    \UnaryInfC{$Γ, ?k:ξ \vdash ?a: \Tint$}
    \rulename{App}
    \BinaryInfC{$Γ, ?k:ξ \vdash ?k~?a: \Tbool → \Tint$}
  \end{prooftree}
  \onslide<2->{
    Das sind zwei Variablentypen, also zwei \textit{Var}-Regeln;
    die linke ist polymorph:
    \begin{prooftree}
      \AxiomC{$(Γ, ?k:ξ)(?k) = ξ$}
      \AxiomC{$ξ \succeq \Tint → \Tbool → \Tint$}
      \rulename{Var}
      \BinaryInfC{$Γ, ?k:ξ \vdash ?k: \Tint → \Tbool → \Tint$}
    \end{prooftree}
  }
  (Erinnerung: $ξ = ∀α. ∀β. α → β → α$)
\end{frame}

\begin{frame}
  \frametitle{Typprüfung}
  Die verbleibende Beweisverpflichtung ist
  \[Γ, ?k:ξ \vdash ?k~?b~?c: \Tbool\]
  \pause
  Zweimal \textit{App}, zweimal \textit{Var}:
  \begin{prooftree}
    \AxiomC{$Γ, ?k:ξ \vdash ?k~?b: \Tchar → \Tbool$}
    \AxiomC{$(Γ, ?k:ξ)(?c) = \Tchar$}
    \rulename{Var}
    \UnaryInfC{$Γ, ?k:ξ \vdash ?c: \Tchar$}
    \rulename{App}
    \BinaryInfC{$Γ, ?k:ξ \vdash ?k~?b~?c: \Tbool$}
  \end{prooftree}
  \begin{prooftree}
    \AxiomC{$Γ, ?k:ξ \vdash ?k: \Tbool → \Tchar → \Tbool$}
    \AxiomC{$(Γ, ?k:ξ)(?b) = \Tbool$}
    \rulename{Var}
    \UnaryInfC{$Γ, ?k:ξ \vdash ?b: \Tbool$}
    \rulename{App}
    \BinaryInfC{$Γ, ?k:ξ \vdash ?k~?b: \Tchar → \Tbool$}
  \end{prooftree}
\end{frame}

\begin{frame}
  \frametitle{Typprüfung}
  Und für den Beweis von
  \[Γ, ?k:ξ \vdash ?k: \Tbool → \Tchar → \Tbool\]
  verwenden wir erneut die polymorphe Variante der \textit{Var}-Regel:
  \begin{prooftree}
    \AxiomC{$(Γ, ?k:ξ)(?k) = ξ$}
    \AxiomC{$ξ \succeq \Tbool → \Tchar → \Tbool$}
    \BinaryInfC{$Γ, ?k:ξ \vdash ?k: \Tbool → \Tchar → \Tbool$}
  \end{prooftree}
  ($ξ$ ist immer noch $∀α. ∀β. α → β → α$.)
\end{frame}

\end{document}
