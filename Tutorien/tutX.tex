\documentclass{beamer}
\usepackage{tut}

\def\tuttitle{Weihnachtsblatt}
\date{2017-01-16/17}

\begin{document}
\normalsize
\normalem

\begin{frame}[plain]
  \titlepage
\end{frame}

\begin{frame}
  \frametitle{Tutorenrätsel}
  Schreibe ein Prolog-Programm, das folgendes Rätsel löst:
  \begin{table}
    \begin{tabular}{|c|c|c|c|c|}\hline
      & Position 1 & Position 2 & Position 3 & Position 4 \\\hline
      Name &&&& \\\hline
      Lieblingsspiel &&&& \\\hline
      programmiert gern in… &&&& \\\hline
      Lieblingsalgorithmus &&&& \\\hline
    \end{tabular}
  \end{table}
  \begin{enumerate}
  \item Der Tutor, dessen Lieblingsalgorithmus Bogosort ist, befindet sich links von dem Tutor, der am liebsten in Ceylon programmiert.
  \item Der Tutor, der gern Antichamber spielt, befindet sich links von dem Tutor, der gern in C\# programmiert.
  \item …
  \end{enumerate}
\end{frame}

\begin{frame}[fragile]
  \prolog
  \frametitle{\lstinline{indexOf}}
  Implementiere Prädikat \lstinline{indexOf},
  wobei \lstinline{indexOf(E, L, N)} genau dann gilt,
  wenn \lstinline{E} in der Liste \lstinline{L} an Stelle \lstinline{N} steht.
  \pause
  \begin{lstlisting}
    indexOf(X, [X|_], 0).
    indexOf(X, [_|Xs], N) :- indexOf(X, Xs, M), N is M+1.
  \end{lstlisting}
\end{frame}

\begin{frame}[fragile]
  \prolog
  \frametitle{\lstinline{solution}}
  Implementiere Prädikat \lstinline{solution},
  das die Lösung des Rätsels berechnet.
  Lösung wird als Liste von Listen
  \begin{lstlisting}
    [[_,_,_,_],[_,_,_,_],[_,_,_,_],[_,_,_,_]]
  \end{lstlisting}
  modelliert,
  mit den Listenelementen Name, Spiel, Programmiersprache, Algorithmus.
  Implementiere jeden Hinweis als eigenes Prädikat.
  \pause
  \begin{lstlisting}
    solution(S) :-
    S = [[_,_,_,_],[_,_,_,_],[_,_,_,_],[_,_,_,_]],
    clue1(S), clue2(S), clue3(S),
    clue4(S), clue5(S), clue6(S),
    clue7(S), clue8(S), clue9(S).
  \end{lstlisting}
\end{frame}

\begin{frame}[fragile]
  \prolog
  \frametitle{\lstinline{clue1}}
  \begin{quote}
    Der Tutor, dessen Lieblingsalgorithmus Bogosort ist, befindet sich links von dem Tutor, der am liebsten in Ceylon programmiert.
  \end{quote}
  \small{(Reihenfolge: Name, Spiel, Programmiersprache, Algorithmus)}
  \pause
  \begin{lstlisting}
    clue1(Solution) :-
      indexOf([_,_,_,bogosort], Solution, N1),
      indexOf([_,_,ceylon,_], Solution, N2),
      N1 < N2.
  \end{lstlisting}
\end{frame}

\begin{frame}[fragile]
  \prolog
  \frametitle{\lstinline{clue2}}
  \begin{quote}
    Der Tutor, der gern Antichamber spielt, befindet sich links von dem Tutor, der gern in C\# programmiert.
  \end{quote}
  \small{(Reihenfolge: Name, Spiel, Programmiersprache, Algorithmus)}
  \pause
  \begin{lstlisting}
    clue2(Solution) :-
      indexOf([_,antichamber,_,_], Solution, N1),
      indexOf([_,_,csharp,_], Solution, N2),
      N1 < N2.
  \end{lstlisting}
\end{frame}

\begin{frame}[fragile]
  \prolog
  \frametitle{\lstinline{clue3}}
  \begin{quote}
    Mindestens ein Tutor steht zwischen Henning und dem Tutor, der gern Donkey Kong Country spielt.    
  \end{quote}
  \small{(Reihenfolge: Name, Spiel, Programmiersprache, Algorithmus)}
  \pause
  \begin{lstlisting}
    clue3(Solution) :-
	  indexOf([henning,_,_,_], Solution, N1),
	  indexOf([_,donkeykongcountry,_,_], Solution, N2),
	  AbsDiff is abs(N2 - N1),
	  AbsDiff >= 2.
  \end{lstlisting}
\end{frame}

\begin{frame}[fragile]
  \prolog
  \frametitle{\lstinline{clue4}}
  \begin{quote}
    Mindestens ein Tutor befindet sich rechts von dem Tutor, der gern Gothic 1 spielt, und gleichzeitig links von dem Tutor, der gern den Boyer-Moore-Algorithmus zum String-Matching verwendet.
  \end{quote}
  \small{(Reihenfolge: Name, Spiel, Programmiersprache, Algorithmus)}
  \pause
  \begin{lstlisting}
    clue4(Solution) :-
	  indexOf([_,gothic,_,_], Solution, N1),
	  indexOf([_,_,_,boyermoore], Solution, N2),
	  Diff is N2 - N1,
	  Diff >= 2.
  \end{lstlisting}
\end{frame}

\begin{frame}[fragile]
  \prolog
  \frametitle{\lstinline{clue5}}
  \begin{quote}
    Der Tutor, der am liebsten in Scala programmiert, steht direkt neben dem Tutor, der gern Donkey Kong Country spielt.
  \end{quote}
  \small{(Reihenfolge: Name, Spiel, Programmiersprache, Algorithmus)}
  \pause
  \begin{lstlisting}
    clue5(Solution) :-
      indexOf([_,_,scala,_], Solution, N1),
      indexOf([_,donkeykongcountry,_,_], Solution, N2),
      AbsDiff is abs(N2 - N1),
      AbsDiff == 1.
  \end{lstlisting}
\end{frame}

\begin{frame}[fragile]
  \prolog
  \frametitle{\lstinline{clue6}}
  \begin{quote}
    Der Tutor, der am liebsten mit dem Tarjan-Algorithmus starke Zusammenhangskomponenten sucht, steht direkt links neben Tobias.
  \end{quote}
  \small{(Reihenfolge: Name, Spiel, Programmiersprache, Algorithmus)}
  \pause
  \begin{lstlisting}
    clue6(Solution) :-
      indexOf([_,_,_,tarjan], Solution, N1),
      indexOf([tobias,_,_,_], Solution, N2),
      N2 is N1 + 1.
  \end{lstlisting}
\end{frame}

\begin{frame}[fragile]
  \prolog
  \frametitle{\lstinline{clue7}}
  \begin{quote}
    Zwischen dem TypeScript-Programmierer und Lucas stehen genau zwei Tutoren.
  \end{quote}
  \small{(Reihenfolge: Name, Spiel, Programmiersprache, Algorithmus)}
  \pause
  \begin{lstlisting}
    clue7(Solution) :-
      indexOf([_,_,typescript,_], Solution, N1),
      indexOf([lucas,_,_,_], Solution, N2),
	  AbsDiff is abs(N2-N1),
	  AbsDiff == 3.
  \end{lstlisting}
\end{frame}

\begin{frame}[fragile]
  \prolog
  \frametitle{\lstinline{clue8}}
  \begin{quote}
    Der Tutor, der direkt rechts neben Jonas steht, spielt gerne Tiny Tiger.
  \end{quote}
  \small{(Reihenfolge: Name, Spiel, Programmiersprache, Algorithmus)}
  \pause
  \begin{lstlisting}
    clue8(Solution) :-
      indexOf([jonas,_,_,_], Solution, N1),
      N2 is N1 + 1,
      indexOf([_,tinytiger,_,_], Solution, N2).
  \end{lstlisting}
\end{frame}

\begin{frame}[fragile]
  \prolog
  \frametitle{\lstinline{clue9}}
  \begin{quote}
    Derjenige Tutor, der am liebsten optimale Züge mit dem Minimax-Verfahren mit Alpha-Beta-Pruning findet, spielt gern Gothic 1.
  \end{quote}
  \small{(Reihenfolge: Name, Spiel, Programmiersprache, Algorithmus)}
  \pause
  \begin{lstlisting}
    clue9(Solution) :-
      member([_,gothic,_,alphabeta], Solution).
  \end{lstlisting}
\end{frame}

\begin{frame}[fragile]
  \frametitle{Musik}
  Wenn wir ein Tonsignal (etwa eine Sinuswelle) gegeben haben,
  können wir daraus eine Sound-Datei produzieren,
  indem wir die Datei abtasten und die einzelnen \emph{Samples} in einer \texttt{.wav}-Datei speichern.
  
  Im vorgegebenen Framework repräsentiert der Datentyp
  \begin{lstlisting}
    data Signal = Signal [Double] deriving (Show, Eq)
  \end{lstlisting}
  einen Strom von Samples (jeweils im Bereich $[-1,1]$).
  Mit der Funktion \lstinline{writeWave} kann das Signal in eine Datei geschrieben werden.
  Die Samplerate von \SI{44100}{\hertz} ist in der Konstante \lstinline{sampleRate :: Double} gespeichert,
  außerdem kann mit einer Funktion \lstinline{time2samples :: Double -> Double} von Sekunden zu Samples konvertiert werden.
\end{frame}

\begin{frame}[fragile]
  \frametitle{\lstinline{constant}, \lstinline{silence}}
  Schreibe zwei Generatoren für unendliche, konstante Signale:
  \begin{lstlisting}
    constant :: Double -> Signal
    silence :: Signal
  \end{lstlisting}
  \pause
  \begin{lstlisting}
    constant x = Signal (repeat x)
    silence = constant 0
  \end{lstlisting}
\end{frame}

\begin{frame}[fragile]
  \frametitle{\lstinline{sine}}
  Schreibe Generator für Sinuswelle der gegebenen Frequenz in Hertz.
  (Anpassung an Samplerate nicht vergessen!)
  \begin{lstlisting}
    sine :: Double -> Signal
  \end{lstlisting}
  \pause
  \begin{lstlisting}
    time2rad f = 2*pi*f/sampleRate
    
    sine f = map scaledSin [0..]
      where scaledSin t = sin (time2rad (t*freq))
  \end{lstlisting}
\end{frame}

\begin{frame}[fragile]
  \frametitle{\lstinline{trim}}
  \lstinline{trim signal t} soll das Signal \lstinline{signal} nach \lstinline{t} Sekunden abschneiden.
  Diese Funktion kann dabei nützlich sein:
  \begin{lstlisting}
    double2int = fromInteger . floor
  \end{lstlisting}
  \pause
  \begin{lstlisting}
    trim (Signal signal) t = Signal (take samples signal)
      where samples = double2int $ time2samples t
  \end{lstlisting}
  Damit können wir einen einsekündigen Kammerton A erzeugen und anhören:
  \begin{lstlisting}
    writeWave "a.wav" (sine 440 `trim` 1.0)
  \end{lstlisting}
\end{frame}

\begin{frame}[fragile]
  \frametitle{\lstinline{Num}, \lstinline{Fractional}}
  Implementiere die Typklassen \lstinline{Num} (\lstinline{+}, \lstinline{-}, \lstinline{*}, \lstinline{abs}, \lstinline{signum}, \lstinline{fromInteger})
  und \lstinline{Fractional} (\lstinline{/}, \lstinline{fromRational}) für \lstinline{Signal},
  damit wir mit Signalen direkt rechnen können.
  \pause
  \begin{lstlisting}
    instance Num Signal where
      (Signal s1) + (Signal s2) = Signal (zipWith (+) s1 s2)
      (Signal s1) - (Signal s2) = Signal (zipWith (-) s1 s2)
      (Signal s1) * (Signal s2) = Signal (zipWith (*) s1 s2)
      abs (Signal signal)    = Signal (map abs signal)
      signum (Signal signal) = Signal (map signum signal)
      fromInteger i          = constant (fromInteger i)
    
    instance Fractional Signal where
      (Signal s1) / (Signal s2) = Signal (zipWith (/) s1 s2)
      fromRational x         = constant (fromRational x)
  \end{lstlisting}
\end{frame}

\begin{frame}[fragile]
  \frametitle{\lstinline{integrate}}
  Um künstliche Instrumente mit \emph{Frequenzmodulation} zu erzeugen,
  brauchen wir zunächst eine Funktion,
  um ein Signal zu integrieren (über Trapezintegration,
  die Höhe eines Abschnitts ist also immer die Mitte der zwei umgebenden Samples).
  \begin{lstlisting}
    integrate :: Signal -> Signal
  \end{lstlisting}
  \pause
  \begin{lstlisting}
    integrate (Signal s) = Signal (integrateSignal 0 0 s)
      where integrateSignal acc y [] = []
            integrateSignal acc y (x:xs) =
              acc : integrateSignal
                    (acc + (x+y)/2 * 1/sampleRate) x xs
  \end{lstlisting}
\end{frame}

\begin{frame}[fragile]
  \frametitle{\lstinline{modulatedSine}}
  Jetzt wollen wir folgendes Signal generieren:
  \[u_t = \sin\left(2 \uppi c \cdot t + c \cdot \int_0^t m_τ \mathrm dτ\right)\]
  Dabei ist $c$ die Trägerfrequenz, auf die das Modulationssignal $m$ addiert wird.
  \pause
  \begin{lstlisting}
    modulatedSine freq s = modul (integrate s)
      where
        modul (Signal s') = Signal (zipWith combine s' [0..])
        combine df t = sin (freq * (time2rad t + df))
  \end{lstlisting}
\end{frame}

\begin{frame}[fragile]
  \frametitle{\lstinline{rampUp}}
  Mit \lstinline{rampUp :: Double -> Signal} soll ein Signal erzeugt werden,
  das in der angegebenen Zeit (in Sekunden) von 0 auf 1 steigt.
  \input{../../programmierparadigmen-tutoren/uebungsaufgaben/Funktional/sound/rampup.tex}
  \pause
  \begin{lstlisting}
    rampUp 0 = Signal [] -- Vermeide Division durch 0 unten
    rampUp t = Signal (map (/samples) [0..samples])
    where samples = time2samples t
  \end{lstlisting}
\end{frame}

\begin{frame}[fragile]
  \frametitle{\lstinline{ramp}}
  \lstinline{ramp from to time} soll als Erweiterung von \lstinline{rampTo} ein Signal generieren,
  das innerhalb von \lstinline{time} vom Level \lstinline{from} zu \lstinline{to} steigt (oder fällt).
  \pause
  \begin{lstlisting}
    ramp from to time =
      (constant from) +
      ((rampUp time) * (constant (to-from)))
  \end{lstlisting}
\end{frame}

\begin{frame}[fragile]
  \frametitle{\lstinline{append}}
  \lstinline{append} soll zwei Signale aneinanderhängen.
  \pause
  \begin{lstlisting}
    append (Signal s1) (Signal s2) = Signal (s1 ++ s2)
  \end{lstlisting}
\end{frame}

\begin{frame}[fragile]
  \frametitle{\lstinline{hullCurve}}
  Jetzt können wir eine Hüllkurve implementieren.
  \lstinline{hullCurve attack decay decayLevel release}
  soll die abgebildete Hüllkurve generieren.
  \input{../../programmierparadigmen-tutoren/uebungsaufgaben/Funktional/sound/hullcurve.tex}
  \pause
  \begin{lstlisting}
    hullCurve attack decay decayLevel release
      = (rampUp attack) `append`
        (ramp 1.0 decayLevel decay) `append`
        (ramp decayLevel 0.0 release)
  \end{lstlisting}
\end{frame}

\begin{frame}[fragile]
  \frametitle{\lstinline{play}}
  Zu guter Letzt soll jetzt noch eine Sequencer-Funktion implementiert werden:
  sie nimmt ein Instrument (macht Frequenz zu Signal)
  und eine Liste von Noten mit Länge entgegen
  und setzt sie zu einem langen Signal zusammen.
  \begin{lstlisting}
    play :: (Double -> Signal) -> [(Double,Double)] -> Signal
  \end{lstlisting}
  \pause
  \begin{lstlisting}
    play instrument notes
      = foldr (append . note) (Signal []) notes
      where note (0,dur) = silence `trim` dur -- ?
            note (f,dur) = ((instrument f) `append` silence)
                           `trim` dur
  \end{lstlisting}
\end{frame}

\end{document}
