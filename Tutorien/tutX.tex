\documentclass{beamer}
\usepackage{tut}

\def\tuttitle{Weihnachtsblatt}
\date{2017-01-16/17}

\begin{document}
\normalsize
\normalem

\begin{frame}[plain]
  \titlepage
\end{frame}

\begin{frame}
  \frametitle{Tutorenrätsel}
  Schreibe ein Prolog-Programm, das folgendes Rätsel löst:
  \begin{table}
    \begin{tabular}{|c|c|c|c|c|}\hline
      & Position 1 & Position 2 & Position 3 & Position 4 \\\hline
      Name &&&& \\\hline
      Lieblingsspiel &&&& \\\hline
      programmiert gern in… &&&& \\\hline
      Lieblingsalgorithmus &&&& \\\hline
    \end{tabular}
  \end{table}
  \begin{enumerate}
  \item Der Tutor, dessen Lieblingsalgorithmus Bogosort ist, befindet sich links von dem Tutor, der am liebsten in Ceylon programmiert.
  \item Der Tutor, der gern Antichamber spielt, befindet sich links von dem Tutor, der gern in C\# programmiert.
  \item …
  \end{enumerate}
\end{frame}

\begin{frame}[fragile]
  \prolog
  \frametitle{\lstinline{indexOf}}
  Implementiere Prädikat \lstinline{indexOf},
  wobei \lstinline{indexOf(E, L, N)} genau dann gilt,
  wenn \lstinline{E} in der Liste \lstinline{L} an Stelle \lstinline{N} steht.
  \pause
  \begin{lstlisting}
    indexOf(X, [X|_], 0).
    indexOf(X, [_|Xs], N) :- indexOf(X, Xs, M), N is M+1.
  \end{lstlisting}
\end{frame}

\begin{frame}[fragile]
  \prolog
  \frametitle{\lstinline{solution}}
  Implementiere Prädikat \lstinline{solution},
  das die Lösung des Rätsels berechnet.
  Lösung wird als Liste von Listen
  \begin{lstlisting}
    [[_,_,_,_],[_,_,_,_],[_,_,_,_],[_,_,_,_]]
  \end{lstlisting}
  modelliert,
  mit den Listenelementen Name, Spiel, Programmiersprache, Algorithmus.
  Implementiere jeden Hinweis als eigenes Prädikat.
  \pause
  \begin{lstlisting}
    solution(S) :-
    S = [[_,_,_,_],[_,_,_,_],[_,_,_,_],[_,_,_,_]],
    clue1(S), clue2(S), clue3(S),
    clue4(S), clue5(S), clue6(S),
    clue7(S), clue8(S), clue9(S).
  \end{lstlisting}
\end{frame}

\begin{frame}[fragile]
  \prolog
  \frametitle{\lstinline{clue1}}
  \begin{quote}
    Der Tutor, dessen Lieblingsalgorithmus Bogosort ist, befindet sich links von dem Tutor, der am liebsten in Ceylon programmiert.
  \end{quote}
  \small{(Reihenfolge: Name, Spiel, Programmiersprache, Algorithmus)}
  \pause
  \begin{lstlisting}
    clue1(Solution) :-
      indexOf([_,_,_,bogosort], Solution, N1),
      indexOf([_,_,ceylon,_], Solution, N2),
      N1 < N2.
  \end{lstlisting}
\end{frame}

\begin{frame}[fragile]
  \prolog
  \frametitle{\lstinline{clue2}}
  \begin{quote}
    Der Tutor, der gern Antichamber spielt, befindet sich links von dem Tutor, der gern in C\# programmiert.
  \end{quote}
  \small{(Reihenfolge: Name, Spiel, Programmiersprache, Algorithmus)}
  \pause
  \begin{lstlisting}
    clue2(Solution) :-
      indexOf([_,antichamber,_,_], Solution, N1),
      indexOf([_,_,csharp,_], Solution, N2),
      N1 < N2.
  \end{lstlisting}
\end{frame}

\begin{frame}[fragile]
  \prolog
  \frametitle{\lstinline{clue3}}
  \begin{quote}
    Mindestens ein Tutor steht zwischen Henning und dem Tutor, der gern Donkey Kong Country spielt.    
  \end{quote}
  \small{(Reihenfolge: Name, Spiel, Programmiersprache, Algorithmus)}
  \pause
  \begin{lstlisting}
    clue3(Solution) :-
	  indexOf([henning,_,_,_], Solution, N1),
	  indexOf([_,donkeykongcountry,_,_], Solution, N2),
	  AbsDiff is abs(N2 - N1),
	  AbsDiff >= 2.
  \end{lstlisting}
\end{frame}

\begin{frame}[fragile]
  \prolog
  \frametitle{\lstinline{clue4}}
  \begin{quote}
    Mindestens ein Tutor befindet sich rechts von dem Tutor, der gern Gothic 1 spielt, und gleichzeitig links von dem Tutor, der gern den Boyer-Moore-Algorithmus zum String-Matching verwendet.
  \end{quote}
  \small{(Reihenfolge: Name, Spiel, Programmiersprache, Algorithmus)}
  \pause
  \begin{lstlisting}
    clue4(Solution) :-
	  indexOf([_,gothic,_,_], Solution, N1),
	  indexOf([_,_,_,boyermoore], Solution, N2),
	  Diff is N2 - N1,
	  Diff >= 2.
  \end{lstlisting}
\end{frame}

\begin{frame}[fragile]
  \prolog
  \frametitle{\lstinline{clue5}}
  \begin{quote}
    Der Tutor, der am liebsten in Scala programmiert, steht direkt neben dem Tutor, der gern Donkey Kong Country spielt.
  \end{quote}
  \small{(Reihenfolge: Name, Spiel, Programmiersprache, Algorithmus)}
  \pause
  \begin{lstlisting}
    clue5(Solution) :-
      indexOf([_,_,scala,_], Solution, N1),
      indexOf([_,donkeykongcountry,_,_], Solution, N2),
      AbsDiff is abs(N2 - N1),
      AbsDiff == 1.
  \end{lstlisting}
\end{frame}

\begin{frame}[fragile]
  \prolog
  \frametitle{\lstinline{clue6}}
  \begin{quote}
    Der Tutor, der am liebsten mit dem Tarjan-Algorithmus starke Zusammenhangskomponenten sucht, steht direkt links neben Tobias.
  \end{quote}
  \small{(Reihenfolge: Name, Spiel, Programmiersprache, Algorithmus)}
  \pause
  \begin{lstlisting}
    clue6(Solution) :-
      indexOf([_,_,_,tarjan], Solution, N1),
      indexOf([tobias,_,_,_], Solution, N2),
      N2 is N1 + 1.
  \end{lstlisting}
\end{frame}

\begin{frame}[fragile]
  \prolog
  \frametitle{\lstinline{clue7}}
  \begin{quote}
    Zwischen dem TypeScript-Programmierer und Lucas stehen genau zwei Tutoren.
  \end{quote}
  \small{(Reihenfolge: Name, Spiel, Programmiersprache, Algorithmus)}
  \pause
  \begin{lstlisting}
    clue7(Solution) :-
      indexOf([_,_,typescript,_], Solution, N1),
      indexOf([lucas,_,_,_], Solution, N2),
	  AbsDiff is abs(N2-N1),
	  AbsDiff == 3.
  \end{lstlisting}
\end{frame}

\begin{frame}[fragile]
  \prolog
  \frametitle{\lstinline{clue8}}
  \begin{quote}
    Der Tutor, der direkt rechts neben Jonas steht, spielt gerne Tiny Tiger.
  \end{quote}
  \small{(Reihenfolge: Name, Spiel, Programmiersprache, Algorithmus)}
  \pause
  \begin{lstlisting}
    clue8(Solution) :-
      indexOf([jonas,_,_,_], Solution, N1),
      N2 is N1 + 1,
      indexOf([_,tinytiger,_,_], Solution, N2).
  \end{lstlisting}
\end{frame}

\begin{frame}[fragile]
  \prolog
  \frametitle{\lstinline{clue9}}
  \begin{quote}
    Derjenige Tutor, der am liebsten optimale Züge mit dem Minimax-Verfahren mit Alpha-Beta-Pruning findet, spielt gern Gothic 1.
  \end{quote}
  \small{(Reihenfolge: Name, Spiel, Programmiersprache, Algorithmus)}
  \pause
  \begin{lstlisting}
    clue9(Solution) :-
      member([_,gothic,_,alphabeta], Solution).
  \end{lstlisting}
\end{frame}

\end{document}
